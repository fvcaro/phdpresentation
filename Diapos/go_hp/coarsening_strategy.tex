\begin{frame}
    \frametitle{A Painless Automatic Adaptive Strategy}
    
    \begin{figure}
        \centering
        \resizebox{0.55\textwidth}{!}{% Resize to the width of the text block
	\begin{tikzpicture}
		\node[rounded rectangle, draw, very thick](Init) {Initial coarse mesh};
		\node[below right=1/2 of Init,rounded rectangle, draw, very thick,color=blue,align=center](ref) {Arbitrary \\(user-defined/global)\\ refinements};
		\node[below= of ref, rounded rectangle, draw, very thick,color=green!50!black, align=center](unrefProcess) {Quasi-optimal\\$hp$-unrefinements};
		\node[below right=1/2 of unrefProcess,rounded rectangle, draw, very thick](final) {Adapted mesh};

		\draw[->, thick] (Init) -| (ref);
		\draw[->, thick,color=blue] (ref.east) to[out=0,in=0] (unrefProcess.east);
		\draw[->, thick,color=green!50!black] (unrefProcess.west) to[out=180,in=180] node[pos=0.5](PinExitRef){} (ref.west);
		\draw[->, thick] (unrefProcess) |- (final);
	\end{tikzpicture}
}
    \end{figure}

    \begin{thebibliography}{10}
        \beamertemplatearticlebibitems
        \scriptsize
        \bibitem{darrigrand2020painless}
        V. Darrigrand, D. Pardo, T. Chaumont-Frelet, I. G{\'o}mez-Revuelto, L. E. Garc{\'i}a-Castillo
        \newblock A painless automatic $hp$-adaptive strategy for elliptic problems
        \newblock \emph{Finite Elements in Analysis and Design}, 2020.

        \beamertemplatearticlebibitems
        \scriptsize
        \bibitem{caro2022painless}
        F. V. Caro, V. Darrigrand, J. Alvarez-Aramberri, E. Alberdi, D. Pardo
        \newblock A painless multi-level automatic goal-oriented $hp$-adaptive coarsening strategy for elliptic and non-elliptic problems
        \newblock \emph{Computer Methods in Applied Mechanics and Engineering}, 2022.
    \end{thebibliography}
\end{frame}

\begin{frame}
    \frametitle{Quasi-optimal \( hp \)-unrefinements}
    \vfill
    \begin{center}
        \begin{figure}
            \centering
        		\begin{tikzpicture}
                \node[rounded rectangle, draw, very thick] (Solve) {Solve};
                \node[right = of Solve, rounded rectangle, draw, very thick, align=center] (Estimate) {Estimate};
                \node[right =1.2 of Estimate, rounded rectangle, draw, very thick] (Mark) {Mark};
                \node[right = of Mark, rounded rectangle, draw, very thick, align=center] (Update) {Update \\ the mesh};
                \draw[->, very thick] (Solve) -- (Estimate);
                \draw[->, very thick] (Estimate) -- (Mark);
                \draw[->, very thick] (Mark) -- (Update);
                
                \node[below= 1/2 of Estimate, align=center, rounded corners, rectangle, draw, very thick, dashed] (DetailsEstimate) {
                    $\bullet$ Find the {\color{green!50!black} removable} workers\\
                    $\bullet$ Estimate their contribution
                };
                \node[below= 2 of Mark, align=center, rounded corners, rectangle, draw, very thick, dashed] (DetailsMark) {
                    $\bullet$ Mark the {\color{green!50!black} lazy} workers
                };
                \node[below= 3 of Update, align=center, rounded corners, rectangle, draw, very thick, dashed] (DetailsUpdate) {
                    $\bullet$ Fire them
                };

                \draw[->, very thick, dashed] (Estimate) -- (DetailsEstimate);
                \draw[->, very thick, dashed] (Mark) -- (DetailsMark);
                \draw[->, very thick, dashed] (Update) -- (DetailsUpdate);
            \end{tikzpicture}
            \caption{Quasi-optimal \( hp \)-unrefinement steps}
        \end{figure}
    \end{center}    
\end{frame}

\begin{frame}
	\frametitle{Removable 1D Basis Functions}
	\begin{block}{Definition}
	We define {\color{green!50!black} removable} basis functions as those we can eliminate from the discretization without modifying any other basis 	function while preserving \textbf{complete} polynomial spaces. 
	\end{block}
	\begin{figure}
	 \hspace{-1cm}
	\centering
	 \begin{subfigure}[b]{\subplotwidth}
	\centering
	\begin{tikzpicture}[x=2cm,y=2cm,decoration={markings,
     		mark=at position 0 with{\draw (0pt,-2pt) -- (0pt,2pt);},
     		mark=at position 1 with{\draw (0pt,-2pt) -- (0pt,2pt);},
	}
	]

	\node(father1) at (0,0) {};
	\node(father2) at (1,0) {};
	\node(father3) at (2,0) {};

	\node(son1) at (0,-1) {};
	\node(son2)  at (0.5,-1) {};
	\node(son3)  at (1,-1) {};
	\node(son4)  at (1.5,-1) {};
	\node(son5)  at (2,-1) {};

	\node(basis1) at ($(father2)+(0,0.5)$){};
	\node(basis2) at ($(son2)+(0,0.5)$){};
	\node(basis3) at ($(son4)+(0,0.5)$){};

	\draw [postaction={decorate}] (father1.center) -- (father2.center) node[pos=0.5](elemfather1){};
	\draw [postaction={decorate}] (father2.center) -- (father3.center) node[pos=0.5](elemfather2){};

	\draw[dashed] (father1.center) -- (son1.center);
	\draw[dashed] (father2.center) -- (son3.center);

	\draw [postaction={decorate}] (son1.center) -- (son2.center) node[pos=0.5](elemson1){};
	\draw [postaction={decorate}] (son2.center) -- (son3.center) node[pos=0.5](elemson2){};

	\node[anchor=north] at (elemfather1) {};
	\node[anchor=north] at (elemfather2) {};

	\node[anchor=north] at (elemson1) {};
	\node[anchor=north] at (elemson2) {};
	% linear basis function
	\draw[color=red, thick] (father1.center) -- (basis1.center) -- (father3.center);
	\draw[color=red, thick] (son1.center) -- (basis2.center) -- (son3.center);
	% degree 2 basis function
	\draw[color=red, thick]    (son1.center) .. controls ($(elemson1)+(-0.1,+0.15)$) and ($(elemson1)+(0.1,0.15)$) ..  (son2.center) ;
	\draw[color=blue,ultra thick]    (son2.center) .. controls ($(elemson2)+(-0.1,+0.15)$) and ($(elemson2)+(0.1,0.15)$) ..  (son3.center);
	\draw[color=blue,ultra thick]    (father2.center) .. controls ($(elemfather2)+(-0.1,+0.15)$) and ($(elemfather2)+(0.1,0.15)$) ..  (father3.center) ;
	% degree 3 basis function
	\draw[color=blue,ultra thick]   (son1.center) .. controls ($(elemson1)+(-0.1,-0.15)$) and ($(elemson1)+(0.1,0.15)$) ..  (son2.center);
	\end{tikzpicture}
	\caption{$hp$-case\label{fig:basis1Dhp}}
	\end{subfigure}
	\begin{subfigure}[b]{\subplotwidth}
	\centering
	\begin{tikzpicture}[x=2cm,y=2cm,decoration={markings,% switch on markings mark=% actually add a mark
     		mark=at position 0 with{\draw (0pt,-2pt) -- (0pt,2pt);},
     	mark=at position 1 with{\draw (0pt,-2pt) -- (0pt,2pt);},
	}
	]
	\node(father1) at (0,0) {};
	\node(father2) at (1,0) {};
	\node(father3) at (2,0) {};

	\node(son1) at (0,-1) {};
	\node(son2)  at (0.5,-1) {};
	\node(son3)  at (1,-1) {};	
	\node(son4)  at (1.5,-1) {};
	\node(son5)  at (2,-1) {};

	\node(basis1) at ($(father2)+(0,0.5)$){};
	\node(basis2) at ($(son2)+(0,0.5)$){};
	\node(basis3) at ($(son4)+(0,0.5)$){};

	\draw [postaction={decorate}] (father1.center) -- (father2.center) node[pos=0.5](elemfather1){};
	\draw [postaction={decorate}] (father2.center) -- (father3.center) node[pos=0.5](elemfather2){};

	\draw[dashed] (father1.center) -- (son1.center);
	\draw[dashed] (father2.center) -- (son3.center);

	\draw [postaction={decorate}] (son1.center) -- (son2.center) node[pos=0.5](elemson1){};
	\draw [postaction={decorate}] (son2.center) -- (son3.center) node[pos=0.5](elemson2){};

	\node[anchor=north] at (elemfather1) {};
	\node[anchor=north] at (elemfather2) {};

	\node[anchor=north] at (elemson1) {};
	\node[anchor=north] at (elemson2) {};
	% linear basis function
	\draw[color=red, thick]   (father1.center) -- (basis1.center) -- (father3.center);
	\draw[color=blue, ultra thick]   (son1.center) -- (basis2.center) -- (son3.center) ;
	\end{tikzpicture}
	\caption{$h$-case \label{fig:basis1Dh}}
\end{subfigure}
	\caption{Removable 1D basis functions}
	\end{figure}
\end{frame}
