\begin{frame}
	\frametitle{Multi-Level Mesh Data Structure 2D}
	\begin{figure}
	\begin{center}
	%\tikzset{/tikz/external/export next=false}
	\begin{tikzpicture}[x=3cm,y=3cm]
	\begin{scope}[%every node/.append style={yslant=0.5,xslant=-1.3},
        yslant=0.5,xslant=-1.3
        ]
	\draw[fill= gray!60!white, line width=1pt] (0,0) rectangle (1.5,1.5);
	\path[fill=white!80!gray, draw] (0.5,0.5) rectangle (1,1);
	\draw[step=1/2, line width=1pt] (0,0) grid (1.5,1.5);

	\foreach \x in {0,0.5,1, 1.5}{
		\foreach \y in {0,0.5,1,1.5}{
			\node[circle, draw, fill=black, inner sep=2pt] at (\x,\y){};
		}
	}
	\node at (0.5,0.5) (root1){};
	\node at (0.5,1) (root2){};
	\node at (1,0.5) (root3){};
	\node at (1,1) (root4){};
	\end{scope}

	\begin{scope}[
	yshift=0.9cm,
	%every node/.append style={yslant=0.5,xslant=-1.3},
        yslant=0.5,xslant=-1.3
        ]
	\node at (0.5,0.5) (son1){};
	\node at (0.5,1) (son2){};
	\node at (1,0.5) (son3){};
	\node at (1,1) (son4){};

	\node at (0.75,0.5) (son5){};
	\node at (0.75,1) (son6){};
	\node at (0.5,0.75) (son7){};
	\node at (1,0.75) (son8){};

	\foreach \i in {1,2,...,4}{
		\draw[dashed] (son\i) -- (root\i);
	}
	
	\path[fill=gray, opacity=0.6] (0.5,0.5) rectangle (1,1);
	\draw[dotted,  line width=1pt] (0.5,0.5) rectangle (1,1);
	\draw[line width=1pt] (son5) -- (son6);
	\draw[line width=1pt] (son7) -- (son8);

	\foreach \xy in { (son5), (son6), (son7) , (son8)}{
		\node[circle, draw, fill=white, inner sep=2pt] at \xy{};
	}
	\node[circle, draw, fill=black, inner sep=2pt] at (0.75,0.75){};

	\path (son3) to node[pos=0.1](pinEdge){} (son4);

	\node[above right =0.2 of pinEdge, anchor=west, inner sep=1pt](legend1){Dirichlet edge};
	\node[above right =0.2 of son8, anchor=south, inner sep=1pt](legend2){Dirichlet node};

	\draw[color=black, thin, shorten <=-3pt,<-] (pinEdge) -- (legend1.south west);
	\draw[color=black, thin, shorten <=-2pt,<-] (son8) -- (legend2.south);
	\end{scope}
	\end{tikzpicture}
	%\missingfigure[figwidth=6cm]{Testing a long text string}
	\end{center}
	\caption{\alert{Multi-level} 2D mesh without constraints on hanging nodes using Dirichlet nodes. The bubble basis functions are at the \alert{lowest level} of each family \label{fig:Multilevel}}
	\end{figure}
	\vfill
	\begin{thebibliography}{10}
	\beamertemplatearticlebibitems
	\scriptsize
	\bibitem{ZanderBog:2015}
	N. Zander, T. Bog, S. Kollmannsberger, D. Schillinger, and E. Rank
	\newblock Multi-level $hp$-adaptivity: high-order mesh adaptivity without the
  	difficulties of constraining hanging nodes
	\newblock \emph{Computational Mechanics}, 55\penalty0 (3):\penalty0 499--517,
  	Mar 2015
	\end{thebibliography}
\end{frame}