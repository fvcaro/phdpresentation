\begin{frame}{Governing PDEs in Geophysics}
\begin{columns}
\begin{column}{0.35\textwidth}
\begin{tikzpicture}[x=0.8cm, y=0.8cm]
%Material   Block A
\draw[very thin,rounded corners=2pt, fill=orange!30!white ] (-1.5,0) rectangle (1.5,1);
\node[color=black] at (0,0.5) {\footnotesize \textcolor{black}{\textbf{\begin{tabular}{c} Physical \\ properties \end{tabular}}}};
%Measurments Block C
\draw[very thin,rounded corners=2pt, fill=orange!30!white ] (-1.5,-8) rectangle (1.5,-7);
\node[color=black] at (0,-7.5) {\footnotesize \textcolor{black}{\textbf{\begin{tabular}{c} Measurements \end{tabular}}}};
%%PDE -- Block U
%\draw[very thin,rounded corners=2pt, fill=black ] (4.5,0) rectangle (5.5,1);
\draw[very thin, fill=blue!40!white] (0,-3.5) ellipse (0.5 and 0.5);
\node[color=black] at (0,-3.5) {\footnotesize \textcolor{black}{\textbf{\begin{tabular}{c} u \end{tabular}}}};
%
%%Define some nodes for origen-destination
\node (Aup) at (0,1) {};
\node (Adw) at (0,0) {};
\node (Ar) at (1.5,0.5) {};
\node (Al) at (-1.5,0.5) {};
%
\node (Cup) at (0,-7) {};
\node (Cdw) at (0,-8) {};
\node (Cr) at (1.5,-7.5) {};
\node (Cl) at (-1.5,-7.5) {};
%
\node (Uup) at (0,-3) {};
\node (Udw) at (0,-4) {};
%%arrows
\draw [<-,very thick, draw = green!40!black, postaction={decorate,decoration={raise=1ex,text along path,text align=center,text={|\color{green!40!black}\sffamily|Inverse Problem}}}] (Ar) to [bend left=35] (Cr);
%
\draw [<-,very thick, draw = red, postaction={decorate,decoration={raise=1ex,text along path,text align=center,text={|\color{red}\sffamily|Forward Problem}}}] (Cl) to [bend left=35] (Al);
%
\draw [<-,draw = black, very thick,] (Uup) to [bend left=0]  (Adw);
%
\draw [<-,draw = black, very thick, ] (Cup) to [bend left=0]  (Udw);
%
\fill[fill=white] (-1.5,-5.2) rectangle (1.5,-5.8);
\node at (0,-5.5) {\footnotesize Post-processing};
\fill[fill=white] (-1.5,-1.2) rectangle (1.5,-1.8);
\node at (0,-1.5) {\footnotesize Solve PDE};
\end{tikzpicture}
\end{column}
%
\begin{column}{0.55\textwidth}
{\large Maxwell's equations}
\vspace{0,15cm}
\begin{equation}
\notag
\left\{
                \begin{array}{lll}
                \vspace{0,2cm}
                  \mathbf{\nabla} \times \mathbf{H} & = (\boldsymbol{\sigma} + j \omega \boldsymbol{\epsilon})\mathbf{E} + \mathbf{J}^{imp} & \textup{Ampère's law,}\\
                  \vspace{0,2cm}
                  \mathbf{\nabla} \times \mathbf{E} & = -j \omega \boldsymbol{\mu}\mathbf{H} + \mathbf{M}^{imp} & \textup{Faraday's law,}\\
                  \mathbf{\nabla} \cdot (\boldsymbol{\epsilon}\mathbf{E}) & = \rho_{e} & \textup{Gauss' law of} \\
                  \vspace{0,2cm}
                  & & \textup{electricity,} \\
                  \mathbf{\nabla} \cdot (\boldsymbol{\mu}\mathbf{H}) & =0 & \textup{Gauss' law of} \\
                 & & \textup{magnetism.}
                \end{array}
              \right.
\end{equation}
\end{column}
\end{columns}
\end{frame}


