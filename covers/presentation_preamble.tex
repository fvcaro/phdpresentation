\documentclass[aspectratio=169]{beamer}

\usecolortheme[RGB={08,164,255}]{structure}
\usetheme[height=8mm]{Rochester}
\setbeamertemplate{items}[ball]
\setbeamertemplate{blocks}[rounded][shadow=true]
\setbeamertemplate{navigation symbols}{}
\usefonttheme{structurebold}
\usepackage[english]{babel}
\usepackage{lmodern}
\usepackage{makeidx}
\usepackage{amsmath}
\usepackage{amsfonts}
\usepackage{graphicx}
\usepackage{rotating} 
\usepackage[intoc]{nomencl} 
\usepackage{amssymb} 
\usepackage{pgfpages}
\usepackage{overpic}
\usepackage{subcaption}
\usepackage{nicefrac}
\usepackage{color, colortbl}
\usepackage{empheq}
\usepackage{tcolorbox}
\usepackage{dsfont}
\usepackage{mathtools,cancel,multicol}
%\usepackage{caption}
%\captionsetup[figure]{labelformat=empty}
\captionsetup[table]{labelformat=empty}
\usepackage{float,tikz,pgfplotstable}
\usetikzlibrary{positioning}
\usetikzlibrary{fit}
% ... (rest of your preamble continues unchanged) ...
\usepackage{xstring}
\usetikzlibrary{decorations.markings}
\usetikzlibrary{shapes.arrows}
\usepgfplotslibrary{fillbetween}
\newlength{\stabfreqwidth}
\newlength{\stabboundwidth}
\newlength{\stabestimlwidth}
\newlength{\stabestimmwidth}
\setlength{\stabfreqwidth}{5cm}
\setlength{\stabboundwidth}{2pt}
\setlength{\stabestimmwidth}{2pt}
\setlength{\stabestimlwidth}{1pt}
%%%%%%%
\usepackage{pgfplots}
\usepackage{tikz-3dplot} %after pgfpages and pgfplots
\usetikzlibrary{decorations.text}
\pgfplotsset{compat=newest}
\usepgfplotslibrary{colorbrewer}
\usetikzlibrary[patterns]
\usetikzlibrary{arrows,positioning,shapes}  
\usetikzlibrary{plotmarks}
\usetikzlibrary{decorations.pathmorphing}
\usetikzlibrary{calc}

\tikzset{
    myarrow/.style={
        draw,
        fill=red,
        single arrow,
        minimum height=4.5ex,
        single arrow head extend=1ex
    }
}
\newcommand{\arrowup}{%
\tikz [baseline=-0.5ex]{\node [myarrow,rotate=90] {};}
}
\newcommand{\arrowdown}{%
\tikz [baseline=-1ex]{\node [myarrow,rotate=-90] {};}
}
\newcommand{\arrowright}{%
\tikz [baseline=-1ex]{\node [myarrow,rotate=0] {};}
}
\usepackage{pifont}
\newcommand{\tickYes}{\textcolor{green!80!blue!}{\ding{51}}}%
\newcommand{\tickNo}{\textcolor{red}{\ding{55}}}%

\newcommand*{\TakeFourierOrnament}[1]{{%
\fontencoding{U}\fontfamily{futs}\selectfont\char#1}}
\newcommand*{\danger}{\textcolor{red}{\TakeFourierOrnament{66}}}

\newlength{\plotwidth}
\newlength{\plotheight}

\newlength{\subplotwidth}
\newlength{\subplotheight}

\newlength{\trisubplotwidth}
\newlength{\trisubplotheight}

%\setlength{\plotwidth}{8cm}
 \setlength{\plotwidth}{0.9\textwidth}
%\setlength{\plotheight}{7cm}
\setlength{\plotheight}{0.85\textheight}

\setlength{\subplotwidth}{0.45\textwidth}
\setlength{\subplotheight}{\textwidth}

\setlength{\trisubplotwidth}{0.3\textwidth}
\setlength{\trisubplotheight}{0.3\textwidth}
%
%
\setbeamertemplate{footline}{%
   \raisebox{5pt}{\makebox[\paperwidth]{\hfill\makebox[10pt]{\tiny\insertframenumber}}}}
%%%% ARROW
\usetikzlibrary{fadings,shapes.arrows,shadows}  
\tikzfading[name=arrowfading, top color=transparent!0, bottom color=transparent!95]
\tikzset{arrowfill/.style={top color=blue!20, bottom color=blue!70, general shadow={fill=black, shadow yshift=-0.8ex, path fading=arrowfading}}}
\tikzset{arrowstyle/.style={draw=blue,arrowfill, single arrow,minimum height=#1, single arrow,
single arrow head extend=.4cm,}}
\newcommand{\tikzfancyarrow}[2][2cm]{\tikz[baseline=-0.5ex]\node [arrowstyle=#1] {#2};}
% Custom commands
\newcommand{\ds}{\displaystyle}
\newcommand{\bs}{\boldsymbol}
\newcommand{\wt}{\widetilde}
\newcommand{\wtb}[1]{\overline{\widetilde{#1}}} 
\newcommand{\wtbF}[1]{\overline{\widetilde{\bsF}}} 
\newcommand{\ol}{\overline}
\newcommand{\tcr}{\textcolor{red}}
\newcommand{\tcw}{\textcolor{white}}
\newcommand{\tcb}{\textcolor{blue}}
\newcommand{\tcc}{\textcolor{cyan}}
\newcommand{\bsmu}{\boldsymbol{\mu}}
\newcommand{\sig}{\sigma}
\newcommand{\bssig}{\boldsymbol{\sigma}}
\newcommand{\bsrho}{\boldsymbol{\rho}}
\newcommand{\eps}{\varepsilon}
\newcommand{\bseps}{\boldsymbol{\varepsilon}}
\newcommand{\bsk}{\boldsymbol{k}}
\newcommand{\bsx}{\boldsymbol{x}}
\newcommand{\bsF}{\boldsymbol{F}}
\newcommand{\bsE}{\boldsymbol{E}}
\newcommand{\bsH}{\boldsymbol{H}}
\newcommand{\bsJ}{\boldsymbol{J}}
\newcommand{\bsM}{\boldsymbol{M}}
\newcommand{\bsnabla}{\boldsymbol{\nabla}}
\newcommand{\dO}{d\Omega}
\newcommand{\p}{\mathfrak{p}}
\newcommand{\f}{\mathfrak{f}}
\newcommand{\Crossred}{$\mathbin{\tikz [x=1.4ex,y=1.4ex,line width=.2ex, red] \draw (0,0) -- (1,1) (0,1) -- (1,0);}$}% 
\newcommand{\Crossblue}{$\mathbin{\tikz [x=1.4ex,y=1.4ex,line width=.2ex, blue] \draw (0,0) -- (1,1) (0,1) -- (1,0);}$}% 
\renewcommand\u{\mathbf{u}}
\newcommand\bv{\mathbf{v}}
\newcommand\g{\mathbf{g}}
\newcommand\e{{\mbox{\boldmath $\varepsilon$}}}
\newcommand\bsi{{\mbox{\boldmath $\sigma$}}}
\newcommand\C{\mathcal{C}}
\newcommand\bdiv{{\mathbf{div}}}
\newcommand\R{\mathbb{R}}
\newcommand\bze{{\mbox{\boldmath $\zeta$}}}
\newcommand\I{\mathbf{I}}
\newcommand\tr{\mathrm{tr}}

\DeclareMathOperator*{\argmin}{arg\,min}

\newlength{\xmax}
\setbeamercovered{transparent}

\usepackage{multirow}

\definecolor{material1}{RGB}{243,217,24}
\definecolor{material2}{RGB}{0,101,19}
\definecolor{material3}{RGB}{59,59,248}
\definecolor{material4}{RGB}{0,225,94}
\definecolor{material5}{RGB}{0,170,255}
\definecolor{material1_3}{RGB}{0,255,223}
\definecolor{material2_3}{RGB}{0,191,255}
\definecolor{material3_3}{RGB}{59,59,248}
\definecolor{myGreen}{RGB}{46,139,87}
\definecolor{myblue}{RGB}{08,164,255}
\definecolor{jon_green}{rgb}{0.0, 0.5, 0.0}

\hypersetup{colorlinks,linkcolor=,urlcolor=myblue}
%this scales is necessary to scale a tikz to textwith
\usepackage{environ}
\newsavebox{\measure@tikzpicture}
\NewEnviron{scaletikzpicturetowidth}[1]{%
  \def\tikz@width{#1}%
  \def\tikzscale{1}\begin{lrbox}{\measure@tikzpicture}%
  \BODY
  \end{lrbox}%
  \pgfmathparse{#1/\wd\measure@tikzpicture}%
  \edef\tikzscale{\pgfmathresult}%
  \BODY
}

   \newif\ifdeveloppath
    \tikzset{/tikz/develop clipping path/.is if=developpath,
      /tikz/develop clipping path=true}

    \newcommand{\clippicture}[2]{
      \begin{tikzpicture}
    % Include the image to determine the size and set up the relative coordinate system. Enclose the \includegraphics in \phantom{} once the clipping path has been set up
    \ifdeveloppath
      \node[anchor=south west,inner sep=0] (image) at (0,0) {\includegraphics#1};
    \else
      \node[anchor=south west,inner sep=0] (image) at (0,0) {\phantom{\includegraphics#1}};
    \fi
    \pgfresetboundingbox
    \begin{scope}[x={(image.south east)},y={(image.north west)}]
      % Draw grid while developing clipping path
      \ifdeveloppath
        \draw[help lines,xstep=.1,ystep=.1] (0,0) grid (1,1);
        \foreach \x in {0,1,...,9} { \node [anchor=north] at (\x/10,0) {0.\x}; }
        \foreach \y in {0,1,...,9} { \node [anchor=east] at (0,\y/10) {0.\y}; }
        \draw[red, ultra thick] #2 -- cycle;
      \else
        % Use the path to clip, include the image
        \path[clip] #2 -- cycle;
        \node[anchor=south west,inner sep=0pt] {\includegraphics#1};
      \fi
    \end{scope}
    \end{tikzpicture}
    }
%make blocks with variable color    
\newenvironment{variableblock}[3]{%
  \setbeamercolor{block body}{#2}
  \setbeamercolor{block title}{#3}
  \begin{block}{#1}}{\end{block}}

%to put transparency in image
\usepackage{transparent}
%to create \mybullet{1}{text}
\newcommand{\mybullet}[2]{
\begin{tikzpicture}
\node (A) at (0,0){};
\shade [shading=ball, ball color=structure]  (A) circle (.2);
\node at (0,0){\small \textbf{\textcolor{white}{#1}}};
\node [right=0.2cm of A]{\large #2};
\end{tikzpicture}
}    
% Carlos Tikz preamble
\usetikzlibrary{arrows,positioning} 
\tikzset{
    %Define standard arrow tip
    >=stealth',
    %Define style for boxes
    punkt/.style={
           rectangle,
           rounded corners,
           draw=black, very thick,
           text width=4em,
           minimum height=2em,
           text centered},
    % Define arrow style
    pil/.style={
           ->,
           thick,
           shorten <=2pt,
           shorten >=2pt,}
}

\tikzset{
  invisible/.style={opacity=0},
  visible on/.style={alt={#1{}{invisible}}},
  alt/.code args={<#1>#2#3}{%
    \alt<#1>{\pgfkeysalso{#2}}{\pgfkeysalso{#3}} % \pgfkeysalso doesn't change the path
  },
}

\newcommand{\roundpic}[4][]{
  \tikz\node [circle, minimum width = #2,
    path picture = {
      \node [#1] at (path picture bounding box.center) {
        \includegraphics[width=#3]{#4}};
    }] {};}
    
\usepackage[ruled, vlined]{algorithm2e}
\SetKwComment{Comment}{/* }{ */}
\SetKwInput{KwOutput}{Output}

\usepackage{booktabs, multirow}
\newlength{\plotwidthmesh}
\setlength{\plotwidthmesh}{0.49\textwidth}
\newcommand{\subplotwidthhp}{0.49\textwidth}

\renewcommand{\H}{\mathbb{H}} % H: Hilbert Spaces
\newcommand{\abs}[1]{\left|#1\right|}
\newcommand{\norm}[1]{\left\|#1\right\|}
\newcommand{\te}{\tilde{\epsilon}}
\newcommand{\calT}{\mathcal{T}}
\newcommand{\grad}{\nabla}
\newcommand{\scalar}[2]{\left<#1\,,#2\right>}
\newcommand{\scalaire}[2]{\left<#1\,,#2\right>}