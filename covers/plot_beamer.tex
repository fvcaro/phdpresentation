% !TEX spellcheck = English
\newcommand*{\PbName}{}
\newcommand*{\GOAPbName}{}
\newcommand*{\EnergyPbName}{}
\newcommand*{\ColName}{}
\newcommand*{\FigurePath}{Figures}
\newcommand*{\DataPath}{}
%
%\tikzset{/tikz/external/export next=false}
%
% ==========
% \directadjointsolution
% ==========
\newcommand*{\directadjointsolution}[5]{%
  \renewcommand*{\DataPath}{\FigurePath/#1}
  \pgfplotsset{%
    colormap={paraview}{rgb=(0.231373, 0.298039, 0.752941) rgb=(0.865003, 0.865003, 0.865003) rgb=(0.705882, 0.0156863, 0.14902)}%
  }%
  \pgfplotsset{colormap name={paraview}}%

  \getelemdbl{\DataPath/#3_#2_ColorbarValues.dat}{min_value}{0}{\scaleMin}
  \getelemdbl{\DataPath/#3_#2_ColorbarValues.dat}{max_value}{0}{\scaleMax}
  \getelemdbl{\DataPath/#3_#2_ColorbarValues.dat}{mean_value}{0}{\scaleMean}

  \begin{tikzpicture}
    \begin{axis}[name=master,
        width=\textwidth,height=\textwidth,
        enlargelimits=false,
        xlabel=$x$,
        ylabel=$y$,
        xlabel near ticks,
        ylabel near ticks,
        colorbar horizontal,
        colorbar style={%
            xtick={\scaleMin,\scaleMean,\scaleMax},
            title={#4 solution},
            at={(master.above north west)},anchor=south west,
            yshift=0.5em,
            title style={
                yshift=2pt
            },
            xticklabel pos=upper,
        },
        point meta min=\scaleMin,
        point meta max=\scaleMax,
        xmin=0, xmax=1,
        ymin=0, ymax=1,
        xtick={0,0.5,1},
        ytick={0,0.5,1},
        tick align=outside,
        axis y line*=#5, % Position of y axis, left or right
        axis x line*=bottom,
        axis line style={draw=none},
        tick style={draw=none},
    ]
      \addplot [forget plot] graphics[xmin=0,xmax=1,ymin=0,ymax=1] {\DataPath/coarse_mesh_fine_#2_#3.png};
    \end{axis}
  \end{tikzpicture}
}
% ==========
% \goasolutions
% ==========
\newcommand*{\goasolutions}[2]{%
  \renewcommand*{\DataPath}{\FigurePath/#1/hp/order_1}
  \findmax{\DataPath/outputs.txt}{iter}{\endIter}

  \FormatIntegerTwoDigit{\endIter}{\endIterformated}
  \FormatIntegerThreeDigit{\endIter}{\endIterformatedthree}
  %
  \begin{subfigure}[t]{0.4\textwidth}
    \centering
    \directadjointsolution{#1/hp/order_1}{\endIterformated}{Direct_solution_#2}{Direct}{left}%
    \caption{Solution to the direct problem.}
    \label{fig:#1_dir}
  \end{subfigure}
  ~
  \begin{subfigure}[t]{0.4\textwidth}
   \centering
   \directadjointsolution{#1/hp/order_1}{\endIterformated}{Adjoint_solution_#2}{Adjoint}{right}%
    \caption{Solution to the adjoint problem.}
    \label{fig:#1_adj}
  \end{subfigure}
}
% ==========
% \plothpmeshes
% ==========
\newcommand{\plothpmeshes}[2][]{
  \renewcommand{\DataPath}{\FigurePath/#2/hp/order_1}
  \findmax{\DataPath/outputs.txt}{iter}{\endIter}
  \FormatIntegerTwoDigit{\endIter}{\endIterformated}
  \FormatIntegerThreeDigit{\endIter}{\endIterformatedthree}

  \only<+>{%
    \begin{figure}
      \centering
      \foreach \i in {x,y}{%
        \begin{subfigure}[t]{0.4\textwidth}
          \centering
          \hpmeshes{#2/hp/order_1}{\endIterformatedthree}{\i}%
          \subcaption{Final $hp$-adapted mesh with polynomial orders in the \i-direction.} 
        \end{subfigure}
      }
      %\caption{Final $hp$-adapted meshes after the adaptivity}
    \end{figure}
  }
}
% ========
% \hpmeshes
% ========
\newcommand{\hpmeshes}[3]{%
  \renewcommand{\DataPath}{\FigurePath/#1}

  \lastelement{\DataPath/outputs.txt}{MaxP}{\maxp}
  \lastelement{\DataPath/outputs.txt}{MinP}{\minp}
  \pgfplotsset{colormap/YlOrRd}
  \begin{tikzpicture}
    \begin{axis}[name=master,
        width=\textwidth,height=\textwidth,
        enlargelimits=false,
        xlabel=$x$,
        ylabel=$y$,
        xlabel near ticks,
        ylabel near ticks,
        axis line style={draw=none},
        tick style={draw=none},
        hide axis,
        colorbar horizontal,
        colorbar style={%
            xtick={1,3,...,\maxp},
            title={Order of approximation},
            at={(master.above north west)},anchor=south west,
            yshift=1em,
            title style={
                yshift=2pt
              },
            xticklabel pos=upper,
          },
        point meta min=1,
        point meta max=\maxp,
      ]
      %
      \ifstrequal{#1}{}{%
        \addplot [forget plot] graphics[xmin=-1,xmax=1,ymin=-1,ymax=1] {\DataPath/mesh_#3_#2.png};
      }{%
        \addplot [forget plot] graphics[xmin=0,xmax=1,ymin=0,ymax=1] {\DataPath/mesh_#3_#2.png};
      }
    \end{axis}
  \end{tikzpicture}
}
% ==========
%\comparerrors
% ==========
\newcommand{\comparerrors}[2][]{
  \begin{subfigure}[t]{0.48\textwidth}
    \centering
    \errorleft[#2]{Lower}
  \end{subfigure}
   ~
 \begin{subfigure}[t]{0.48\textwidth}
    \centering
    \errorright[#2]{Error}
  \end{subfigure}
}
% ======
%\errorleft
% ======
\newcommand{\errorleft}[2][]{
  \renewcommand{\DataPath}{\FigurePath/#1}
  \renewcommand{\PbName}{#1} 
  \renewcommand{\ColName}{#2}

  \begin{tikzpicture}
    \pgfplotsset{xmode=log}

    \begin{axis}[name=mainerrorplot,
        xlabel={nDoFs (log scale)},
        ylabel=Relative error in \% (log scale),
        ymode=log,
        xmode=log,
        width=\textwidth,height=\plotheight,
        ylabel near ticks,
        xlabel near ticks,
        enlargelimits=true,
        legend style={draw=black,fill=white,legend cell align=left, at={(0.5,1.01)}, anchor=south},
        legend columns=-1
      ]
        %
        % Plot for hp
	\pgfplotstableread{\FigurePath/\PbName/hp/order_1/outputs.txt}\loadedtable
	\addplot+[line width=1pt] table[x expr=\thisrow{nr_dof}, y expr=\thisrow{\ColName}] {\loadedtable}
	node[pos=0.75, pin={[pin edge=solid, pin distance=0.4cm]270:$hp$}] {}; 

	% Plot for h with p=1
	\pgfplotstableread{\FigurePath/\PbName/h/order_1/outputs.txt}\loadedtable
	\addplot+[line width=1pt] table[x expr=\thisrow{nr_dof}, y expr=\thisrow{\ColName}] {\loadedtable}
	node[pos=0.8, pin={[pin edge=solid, pin distance=0.5cm]90:$h$ ($p=1$)}] {};

	% Plot for h with p=2
	\pgfplotstableread{\FigurePath/\PbName/h/order_2/outputs.txt}\loadedtable
	\addplot+[line width=1pt] table[x expr=\thisrow{nr_dof}, y expr=\thisrow{\ColName}] {\loadedtable}
	node[pos=0.05, pin={[pin edge=solid, pin distance=0.4cm]0:$h$ ($p=2$)}] {};
     
    \end{axis}
  \end{tikzpicture}
   {\caption{Evolution of $\tilde{e}_{\textrm{rel}}^{\textrm{\, energy}}$ in the process.}}
}
% =======
%\errorright
% =======
\newcommand{\errorright}[2][]{
  \renewcommand{\DataPath}{\FigurePath/#1}
  \renewcommand{\PbName}{#1} 
  \renewcommand{\ColName}{#2}

  \begin{tikzpicture}
    \pgfplotsset{xmode=log}

    \begin{axis}[name=mainerrorplot,
        xlabel={nDoFs (log scale)},
        ymode=log,
        xmode=log,
        yticklabel pos=right,
        width=\textwidth,height=\plotheight,
        ylabel near ticks,
        xlabel near ticks,
        enlargelimits=true,
        legend style={draw=black,fill=white,legend cell align=left, at={(0.5,1.01)}, anchor=south},
        legend columns=-1
      ]
      %
      % Plot for hp
\pgfplotstableread{\FigurePath/\PbName/hp/order_1/outputs.txt}\loadedtable
\addplot+[line width=1pt] table[x expr=\thisrow{nr_dof}, y expr=\thisrow{\ColName}] {\loadedtable}
node[pos=0.95, pin={[pin edge=solid, pin distance=0.4cm]90:$hp$}] {}; 

% Plot for h with p=1
\pgfplotstableread{\FigurePath/\PbName/h/order_1/outputs.txt}\loadedtable
\addplot+[line width=1pt] table[x expr=\thisrow{nr_dof}, y expr=\thisrow{\ColName}] {\loadedtable}
node[pos=0.5, pin={[pin edge=solid, pin distance=0.1cm]90:$h$ ($p=1$)}] {};

% Plot for h with p=2
\pgfplotstableread{\FigurePath/\PbName/h/order_2/outputs.txt}\loadedtable
\addplot+[line width=1pt] table[x expr=\thisrow{nr_dof}, y expr=\thisrow{\ColName}] {\loadedtable}
node[pos=0.1, pin={[pin edge=solid, pin distance=0.4cm]270:$h$ ($p=2$)}] {};


    \end{axis}
  \end{tikzpicture}
  {\caption{Evolution of $e_{\textrm{rel}}^{\textrm{QoI}}$ in the process.}}
}

\tikzset{mark size=3}
% =========
% \plothpunref
% =========
\newcommand{\plothpunref}[1]{
    \renewcommand{\DataPath}{\FigurePath/Unref/#1/hp/order_1}
    \findmax{\DataPath/outputs.txt}{iter}{\endIter}

    \findmax{\FigurePath/#1/hp/outputs.txt}{iter}{\endIterh}

    \FormatIntegerTwoDigit{\endIterh}{\endIterhformated}

    \begin{center}
        \begin{figure}
            \foreach \i in {1,2,...,\endIter}{%
                \only<+>{%
                    \begin{subfigure}{\subplotwidthhp}
                        \centering
                        \FormatIntegerThreeDigit{\i}{\iformat}
                        \plothpmeshunref{Unref/#1/hp/order_1}{\iformat}{#1/hp/coarse_mesh_fine_\endIterhformated_Direct_solution_abs} % custom command
                        \subcaption{Adaptive iteration \i ~of \endIter}
                    \end{subfigure}
                    \begin{subfigure}{\subplotwidthhp}
                        \centering
                        % % %
                        \errorplotlog{Unref/#1/hp/order_1}{\i} % custom command
                        \subcaption{ Evolution of the relative error}
                    \end{subfigure}
                    \only<1->{\caption{Adaptive process for a Helmholtz problem}}
                }%
            }
        \end{figure}
    \end{center}
}
% =============
% \plothpmeshunref
% =============
\newcommand{\plothpmeshunref}[3]{%
    \renewcommand{\DataPath}{\FigurePath/#1}

    \begin{tikzpicture}
        \begin{axis}[
            name=master,
            width=\textwidth,
            height=\textwidth,
            enlargelimits=false,
            xlabel=$x$,
            ylabel=$y$,
            xlabel near ticks,
            ylabel near ticks,
            hide axis,
        ]
        %

        \addplot [forget plot] graphics[xmin=0,xmax=1,ymin=0,ymax=1] {\DataPath/mesh_X_#2.png};
        \addplot [forget plot, opacity=0.2] graphics[xmin=0,xmax=1,ymin=0,ymax=1] {\FigurePath/#3};

        \end{axis}
    \end{tikzpicture}%
}

\pgfplotsset{select coords between index/.style 2 args={
    x filter/.code={
        \ifnum\coordindex<#1\def\pgfmathresult{}\fi
        \ifnum\coordindex>#2\def\pgfmathresult{}\fi
    }
}}
% =========
% \errorplotlog
% =========
\newcommand{\errorplotlog}[2]{
    \renewcommand{\DataPath}{\FigurePath/#1}

    \findmax{\DataPath/outputs.txt}{nr_dof}{\maxdof}
    \findmin{\DataPath/outputs.txt}{nr_dof}{\mindof}
    \findmaxDbl{\DataPath/outputs.txt}{Error}{\maxerror}
    \findminDbl{\DataPath/outputs.txt}{Error}{\minerror}

    \begin{tikzpicture}
        %\pgfplotsset{ log y ticks with fixed point,}
        \begin{axis}[
            name=mainerrorplot,
            xlabel={Number of DoF},
            ylabel=Relative error in the QoI (\%),
            ymode=log,
            xmode=log,
            ymin=1e-7,
            ymax=1e3,
            xmin=\mindof,
            xmax=\maxdof,
            width=0.9\textwidth,
            height=0.85\plotwidthmesh,
            ylabel near ticks,
            xlabel near ticks,
            enlargelimits=true,
            legend style={
                draw=black,
                fill=white,
                legend cell align=left,
                at={(0.5,1.01)},
                anchor=south
            },
            legend columns=-1
        ]

        \addplot+[
            line width=1pt,
            select coords between index={0}{#2}
        ] table[
            x expr=\thisrow{nr_dof},
            y expr=\thisrow{Error}
        ] {\FigurePath/#1/outputs.txt};

        \end{axis}
    \end{tikzpicture}
}
% ============
%\compareconver
% ============
\newcommand{\compareconver}[3][]{
  \begin{subfigure}[t]{0.48\textwidth}
    \centering
    \converleft[#1]{#2}{#3}{Error}
  \end{subfigure}
  ~
 \begin{subfigure}[t]{0.48\textwidth}
    \centering
    \converright[#1]{#2}{#3}{Lower}
  \end{subfigure}
}
% ========
%\converleft
% ========
\newcommand{\converleft}[4][]{
  \renewcommand{\DataPath}{\FigurePath/#2}
  \renewcommand{\GOAPbName}{#2} 
  \renewcommand{\EnergyPbName}{#3} 
  \renewcommand{\ColName}{#4}

  \begin{tikzpicture}
    \pgfplotsset{xmode=log}
    
    \begin{axis}[name=mainerrorplot,
          xlabel={Number of DoF},
          ylabel=Relative error in \% (log scale),
          ymode=log,
          xmode=log,
           width=\textwidth,height=\plotheight,
          ylabel near ticks,
          xlabel near ticks,
          enlargelimits=true,
          legend style={draw=black,fill=white,legend cell align=left, at={(0.5,1.01)}, anchor=south},
          legend columns=-1
        ]
        %
        \foreach\method in {#1}{												
          \ifthenelse{\equal{\method}{hp}}{
            \foreach \order in {1}{
                \edef\temp{\noexpand\addlegendentry{GOA}}
                \pgfplotstableread{\FigurePath/\GOAPbName/hp/order_\order/outputs.txt}\loadedtable
                \addplot+[ line width=1pt] table[x expr=\thisrow{nr_dof},y expr=\thisrow{\ColName}] {\loadedtable};
                \temp
                
                \edef\temp{\noexpand\addlegendentry{energy-norm}}
                \pgfplotstableread{\FigurePath/\EnergyPbName/hp/order_\order/outputs.txt}\loadedtable
                \addplot+[ line width=1pt] table[x expr=\thisrow{nr_dof},y expr=\thisrow{\ColName}] {\loadedtable};
                \temp
              }
          }{}
        }
      \end{axis}
  \end{tikzpicture}
  \caption{Evolution of goal-oriented adaptivity.}
  \label{subfig:compagoal}
}
% =========
%\converright
% =========
\newcommand{\converright}[4][]{
  \renewcommand{\DataPath}{\FigurePath/#2}
  \renewcommand{\GOAPbName}{#2} 
  \renewcommand{\EnergyPbName}{#3} 
  \renewcommand{\ColName}{#4}
  
  \begin{tikzpicture}
    \pgfplotsset{xmode=log}
    
    \begin{axis}[name=mainerrorplot,
          xlabel={Number of DoF},
          ymode=log,
          xmode=log,
          yticklabel pos=right,
          width=\textwidth,height=\plotheight,
          ylabel near ticks,
          xlabel near ticks,
          enlargelimits=true,
          legend style={draw=black,fill=white,legend cell align=left, at={(0.5,1.01)}, anchor=south},
          legend columns=-1
        ]
        %
 	\foreach\method in {#1}{												
          \ifthenelse{\equal{\method}{hp}}{
            \foreach \order in {1}{
                \edef\temp{\noexpand\addlegendentry{GOA}}
                \pgfplotstableread{\FigurePath/\GOAPbName/hp/order_\order/outputs.txt}\loadedtable
                \addplot+[ line width=1pt] table[x expr=\thisrow{nr_dof},y expr=\thisrow{\ColName}] {\loadedtable};
                \temp
                
                 \edef\temp{\noexpand\addlegendentry{energy-norm}}
                \pgfplotstableread{\FigurePath/\EnergyPbName/hp/order_\order/outputs.txt}\loadedtable
                \addplot+[ line width=1pt] table[x expr=\thisrow{nr_dof},y expr=\thisrow{\ColName}] {\loadedtable};
                \temp
              }
          }{}
        }
      \end{axis}
  \end{tikzpicture}
  \caption{Evolution of energy-norm adaptivity.}
  \label{subfig:compaenergy}
}
% =========
%\plotsolution
% =========
\newcommand{\plotsolution}[2][]{
  \renewcommand{\DataPath}{\FigurePath/#2/hp/order_1}
  \findmax{\DataPath/outputs.txt}{iter}{\endIter}

  \FormatIntegerTwoDigit{\endIter}{\endIterformated}
  \FormatIntegerThreeDigit{\endIter}{\endIterformatedthree}
  \begin{center}
  \begin{subfigure}[t]{0.425\textwidth}
  	\energysolution{#2/hp/order_1}{\endIterformated}{Direct_solution_real}{#1}%
	\caption{Solution of the example 1.}
  \end{subfigure}
  \end{center}
}
% ===========
%\energysolution
% ===========
\newcommand{\energysolution}[4]{
  \renewcommand{\DataPath}{\FigurePath/#1}
  
  \pgfplotsset{
      colormap={paraview}{rgb=(0.231373, 0.298039, 0.752941) rgb=(0.865003, 0.865003, 0.865003) rgb=(0.705882, 0.0156863, 0.14902)}%
    }
    \pgfplotsset{colormap name={paraview}}

    \getelemdbl{\DataPath/#3_#2_ColorbarValues.dat}{min_value}{0}{\scaleMin}
    \getelemdbl{\DataPath/#3_#2_ColorbarValues.dat}{max_value}{0}{\scaleMax}
    \getelemdbl{\DataPath/#3_#2_ColorbarValues.dat}{mean_value}{0}{\scaleMean}

    \begin{tikzpicture}
      \begin{axis}[name=master,
          width=\textwidth,height=\textwidth,
          enlargelimits=false,
          xlabel=$x$,
          ylabel=$y$,
          xlabel near ticks,
          ylabel near ticks,
          xtick={0,0.5,1},
          ytick={0,0.5,1},
          colorbar horizontal,
          colorbar style={
              xtick={\scaleMin,\scaleMean,\scaleMax},
              title={Value of the solution},
              at={(master.above north west)},anchor=south west,
              yshift=2pt,
              title style={
                  yshift=5pt
                },
              yshift=0.2em,
              xticklabel pos=upper,
            },
          point meta min=\scaleMin,
          point meta max=\scaleMax,
        ]
        \addplot [forget plot] graphics[xmin=0,xmax=1,ymin=0,ymax=1] {\DataPath/coarse_mesh_fine_#2_#3.png};
      \end{axis}
    \end{tikzpicture}
}
% ==========
%\plotmaterials
% ==========
\newcommand{\plotmaterials}[2][]{
  \renewcommand{\DataPath}{\FigurePath/#2/hp/order_1}
  \findmax{\DataPath/outputs.txt}{iter}{\endIter}

  \FormatIntegerTwoDigit{\endIter}{\endIterformated}
  \FormatIntegerThreeDigit{\endIter}{\endIterformatedthree}
  \begin{center}
  	\materials{#2/hp/order_1}{\endIterformated}{materials}{#1}%
  \end{center}
}
% =======
%\materials
% =======
\newcommand{\materials}[4]{
  \renewcommand{\DataPath}{\FigurePath/#1}

    \pgfplotsset{
      colormap={paraview}{rgb=(0.05, 1.0, 1.0) rgb=(1.0, 0.0, 0.0) rgb=(0.0, 1.0, 0.0) rgb=(0.0, 0.0, 1.0)}%
    }
    \pgfplotsset{colormap name={paraview}}%
    \findmax{\DataPath/#3.txt}{uno}{\matuno}
    \findmax{\DataPath/#3.txt}{dos}{\matdos}
    \findmax{\DataPath/#3.txt}{tres}{\mattres}
    \findmax{\DataPath/#3.txt}{cuatro}{\matcuatro}

    \begin{tikzpicture}
      \begin{axis}[name=master,
          width=\textwidth*0.5,height=\textwidth*0.5,
          enlargelimits=false,
          axis line style={draw=none},
          tick style={draw=none},
          xticklabels={},
          yticklabels={},
          colorbar horizontal,
          colorbar sampled,
          colormap access=piecewise constant,
          colorbar style={
              samples=4,
              xtick=data,
              xticklabels={0.01,1,10,1000},
              xticklabel style={xshift=\textwidth*0.5/10},
              title={Diffusivity of the materials},
              at={(master.above north west)},anchor=south west,
              yshift=5pt,
              title style={
                  yshift=5pt
                },
              yshift=0.2em,
              xticklabel pos=upper,
            },
        ]
        %
        \addplot [forget plot] graphics[xmin=0,xmax=1,ymin=0,ymax=1] {\DataPath/#3_#2.png};
      \end{axis}
    \end{tikzpicture}
}
%%%%%%%%%%%%%%%%%%%%%%%%%%%%%%%%%%%%%%%%%%%%%%%%%%%%%%%% OLD
% ==============
% elemental routines
% ==============
\newcommand{\FormatIntegerTwoDigit}[2]{
  \pgfmathsetbasenumberlength{2}
  \pgfmathbasetodec#2{#1}{10}
}

\newcommand{\FormatIntegerThreeDigit}[2]{
  \pgfmathsetbasenumberlength{3}
  \pgfmathbasetodec#2{#1}{10}
}

\newcommand{\getelem}[4]{
  \pgfplotstablegetelem{#3}{#2}\of{#1}%
  \pgfmathtruncatemacro#4{\pgfplotsretval}
  \pgfplotstableclear{\datatable}
}


\newcommand{\getelemdbl}[4]{
  \pgfplotstablegetelem{#3}{#2}\of{#1}%
  \pgfmathsetmacro#4{\pgfplotsretval}
  \pgfplotstableclear{\datatable}
}

\newcommand{\findmax}[3]{
  \pgfplotstableread{#1}{\datatable}
  \pgfplotstablesort[sort key={#2},sort cmp={float >}]{\sorted}{\datatable}%
  \pgfplotstablegetelem{0}{#2}\of{\sorted}%
  \pgfmathtruncatemacro#3{\pgfplotsretval}
  \pgfplotstableclear{\datatable}
}

\newcommand{\findmin}[3]{
  \pgfplotstableread{#1}{\datatable}
  \pgfplotstablesort[sort key={#2},sort cmp={float <}]{\sorted}{\datatable}%
  \pgfplotstablegetelem{0}{#2}\of{\sorted}%
  \pgfmathtruncatemacro#3{\pgfplotsretval}
  \pgfplotstableclear{\datatable}
}

\newcommand{\findmaxDbl}[3]{
  \pgfplotstableread{#1}{\datatable}
  \pgfplotstablesort[sort key={#2},sort cmp={float >}]{\sorted}{\datatable}%
  \pgfplotstablegetelem{0}{#2}\of{\sorted}%
  \pgfmathsetmacro#3{\pgfplotsretval}
  \pgfplotstableclear{\datatable}
}

\newcommand{\lastelement}[3]{
  \pgfplotstableread{#1}{\datatable}
  \pgfplotstablegetrowsof{\datatable} 
  \pgfmathtruncatemacro{\rows}{\pgfplotsretval}
  \pgfmathtruncatemacro{\lastrow}{\rows-1}
  \pgfplotstablegetelem{\lastrow}{#2}\of{\datatable}
  \pgfmathsetmacro#3{\pgfplotsretval}
  \let#3=\pgfplotsretval
  \pgfplotstableclear{\datatable}
}

\newcommand{\findminDbl}[3]{
  \pgfplotstableread{#1}{\datatable}
  \pgfplotstablesort[sort key={#2},sort cmp={float <}]{\sorted}{\datatable}%
  \pgfplotstablegetelem{0}{#2}\of{\sorted}%
  \pgfmathsetmacro#3{\pgfplotsretval}
  \let#3=\pgfplotsretval
  \pgfplotstableclear{\datatable}
}

\pgfplotsset{
  log x ticks with fixed point/.style={
      xticklabel={
          \pgfkeys{/pgf/fpu=true}
          \pgfmathparse{exp(\tick)}%
          \pgfmathprintnumber[fixed relative, precision=3]{\pgfmathresult}
          \pgfkeys{/pgf/fpu=false}
        }
    },
  log y ticks with fixed point/.style={
      yticklabel={
          \pgfkeys{/pgf/fpu=true}
          \pgfmathparse{exp(\tick)}%
          \pgfmathprintnumber[fixed relative, precision=3]{\pgfmathresult}
          \pgfkeys{/pgf/fpu=false}
        }
    }
}

\pgfplotscreateplotcyclelist{custom list style}{%
  color=red, solid, every mark/.append style={solid},mark size=4.0pt,mark=o\\%
  color=blue, solid, every mark/.append style={solid},mark size=3.0pt,mark=square\\%
  color=gray!20!black!40, solid, every mark/.append style={solid}, mark size=2.5pt, mark=triangle\\%
  color=black, solid, every mark/.append style={solid},mark=diamond*\\%
  color=black!50!red, solid, every mark/.append style={solid}, mark=otimes*\\%
  color=black!50!yellow, loosely dashed, every mark/.append style={solid},mark=*\\%
  color=black!50!blue,densely dashed, every mark/.append style={solid},mark=square*\\%
  color=black!50!magenta,dashdotted, every mark/.append style={solid},mark=otimes*\\%
  color=black!50!orange,dashdotdotted, every mark/.append style={solid},mark=star\\%
  color=black!50!cyan, densely dashdotted,every mark/.append style={solid},mark=diamond*\\%
}

\pgfplotsset{%
  every axis plot/.append style= {line width=2pt},
  cycle list name=custom list style,
}
