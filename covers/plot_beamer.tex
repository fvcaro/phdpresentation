% !TEX spellcheck = English
\newcommand*{\PbName}{}
\newcommand*{\GOAPbName}{}
\newcommand*{\EnergyPbName}{}
\newcommand*{\ColName}{}
\newcommand*{\FigurePath}{Figures}
\newcommand*{\DataPath}{}

\newcommand*{\directadjointsolution}[5]{%
  \renewcommand*{\DataPath}{\FigurePath/#1}
  \pgfplotsset{%
    colormap={paraview}{rgb=(0.231373, 0.298039, 0.752941) rgb=(0.865003, 0.865003, 0.865003) rgb=(0.705882, 0.0156863, 0.14902)}%
  }%
  \pgfplotsset{colormap name={paraview}}%

  \getelemdbl{\DataPath/#3_#2_ColorbarValues.dat}{min_value}{0}{\scaleMin}
  \getelemdbl{\DataPath/#3_#2_ColorbarValues.dat}{max_value}{0}{\scaleMax}
  \getelemdbl{\DataPath/#3_#2_ColorbarValues.dat}{mean_value}{0}{\scaleMean}

  %\tikzset{/tikz/external/export next=false}

  \begin{tikzpicture}
    \begin{axis}[name=master,
        width=\textwidth,height=\textwidth,
        enlargelimits=false,
        xlabel=$x$,
        ylabel=$y$,
        xlabel near ticks,
        ylabel near ticks,
        colorbar horizontal,
        colorbar style={%
            xtick={\scaleMin,\scaleMean,\scaleMax},
            title={#4 solution},
            at={(master.above north west)},anchor=south west,
            yshift=2pt,
            title style={yshift=5pt},
            xticklabel pos=upper,
        },
        point meta min=\scaleMin,
        point meta max=\scaleMax,
        xmin=0, xmax=1,
        ymin=0, ymax=1,
        xtick={0,0.5,1},
        ytick={0,0.5,1},
        tick align=outside,
        axis y line*=#5, % Position of y axis, left or right
        axis x line*=bottom,
        axis line style={draw=none},
        tick style={draw=none},
    ]
      \addplot [forget plot] graphics[xmin=0,xmax=1,ymin=0,ymax=1] {\DataPath/coarse_mesh_fine_#2_#3.png};
    \end{axis}
  \end{tikzpicture}
}

\newcommand*{\goasolutions}[2]{%
  \renewcommand*{\DataPath}{\FigurePath/#1/hp/order_1}
  \findmax{\DataPath/outputs.txt}{iter}{\endIter}

  \FormatIntegerTwoDigit{\endIter}{\endIterformated}
  \FormatIntegerThreeDigit{\endIter}{\endIterformatedthree}
  %
  \begin{subfigure}[t]{0.4\textwidth}
    \centering
    \directadjointsolution{#1/hp/order_1}{\endIterformated}{Direct_solution_#2}{Direct}{left}%
    \caption{Solution to the direct problem.}
    \label{fig:#1_dir}
  \end{subfigure}
  ~
  \begin{subfigure}[t]{0.4\textwidth}
   \centering
   \directadjointsolution{#1/hp/order_1}{\endIterformated}{Adjoint_solution_#2}{Adjoint}{right}%
    \caption{Solution to the adjoint problem.}
    \label{fig:#1_adj}
  \end{subfigure}
}

\newcommand{\plothpmeshes}[2][]{
  \renewcommand{\DataPath}{\FigurePath/#2/hp/order_1}
  \findmax{\DataPath/outputs.txt}{iter}{\endIter}
  \FormatIntegerTwoDigit{\endIter}{\endIterformated}
  \FormatIntegerThreeDigit{\endIter}{\endIterformatedthree}

  \only<+>{%
    \begin{figure}
      \centering
      \foreach \i in {x,y}{%
        \begin{subfigure}[t]{0.4\textwidth}
          \centering
          \hpmeshes{#2/hp/order_1}{\endIterformatedthree}{\i}%
          \subcaption{Polynomial orders $p$ in the \i-direction}
        \end{subfigure}
      }
      \caption{Final $hp$-adapted meshes after the adaptivity}
    \end{figure}
  }
}

\newcommand{\hpmeshes}[3]{%
  \renewcommand{\DataPath}{\FigurePath/#1}

  \lastelement{\DataPath/outputs.txt}{MaxP}{\maxp}
  \lastelement{\DataPath/outputs.txt}{MinP}{\minp}
  %\tikzset{/tikz/external/export next=false}
  \pgfplotsset{colormap/YlOrRd}
  \begin{tikzpicture}
    \begin{axis}[name=master,
        width=\textwidth,height=\textwidth,
        enlargelimits=false,
        xlabel=$x$,
        ylabel=$y$,
        xlabel near ticks,
        ylabel near ticks,
        axis line style={draw=none},
        tick style={draw=none},
        hide axis,
        colorbar horizontal,
        colorbar style={%
            xtick={1,3,...,\maxp},
            title={Order of approximation},
            at={(master.above north west)},anchor=south west,
            yshift=1em,
            title style={
                yshift=2pt
              },
            xticklabel pos=upper,
          },
        point meta min=1,
        point meta max=\maxp,
      ]
      %
      \ifstrequal{#1}{}{%
        \addplot [forget plot] graphics[xmin=-1,xmax=1,ymin=-1,ymax=1] {\DataPath/mesh_#3_#2.png};
      }{%
        \addplot [forget plot] graphics[xmin=0,xmax=1,ymin=0,ymax=1] {\DataPath/mesh_#3_#2.png};
      }
    \end{axis}
  \end{tikzpicture}
}
% ========
% unrefmago
% ========
%\newcommand{\plothpunrefmago}[1]{
%  \renewcommand{\DataPath}{\FigurePath/Unref/#1/hp/order_1}
%  \findmax{\DataPath/outputs.txt}{iter}{\endIter}
%
%  \findmax{\FigurePath/#1/hp/outputs.txt}{iter}{\endIterh}
%
%  \FormatIntegerTwoDigit{\endIterh}{\endIterhformated}
%
%  \begin{center}
%    \begin{figure}
%      \foreach \i in {1,2,...,\endIter}{%
%        \only<+>{%
%          \begin{subfigure}{\subplotwidthhp}
%            \centering
%            \tikzset{/tikz/external/export next=false}
%            \FormatIntegerThreeDigit{\i}{\iformat}
%            \plothpmagomesh{Unref/#1/hp/order_1}{\iformat}{#1/hp/coarse_mesh_fine_\endIterhformated_Direct_solution_abs}%
%            \subcaption{Adaptive iteration \i ~of \endIter}
%          \end{subfigure}
%          \begin{subfigure}{\subplotwidthhp}
%            \centering
%            \errorplotlogUnrefBorrar{Unref/#1/hp/order_1}{\i}
%            \subcaption{Evolution of the relative error}
%          \end{subfigure}
%          \only<1->{\caption{Adaptive process for a Helmholtz problem}}
%        }
%      }
%    \end{figure}
%  \end{center}
%}
%
%\newcommand{\plothpmagomesh}[3]{%
%  \renewcommand{\DataPath}{\FigurePath/#1}
%
%  \begin{tikzpicture}
%    \begin{axis}[name=master,
%        width=\textwidth,height=\textwidth,
%        enlargelimits=false,
%        xlabel=$x$,
%        ylabel=$y$,
%        xlabel near ticks,
%        ylabel near ticks,
%        hide axis,
%      ]
%
%      \IfSubStr{#1}{square}{%
%        \addplot [forget plot] graphics[xmin=-1,xmax=1,ymin=-1,ymax=1] {\DataPath/mesh_X_#2.png};
%      }{%
%        \addplot [forget plot] graphics[xmin=0,xmax=1,ymin=0,ymax=1] {\DataPath/mesh_X_#2.png};
%      }
%      \addplot [forget plot, opacity=0.2] graphics[xmin=0,xmax=1,ymin=0,ymax=1] {\FigurePath/#3};
%
%    \end{axis}
%  \end{tikzpicture}%
%}
%
%\newcommand{\errorplotlogUnrefBorrar}[2]{
%  \renewcommand{\DataPath}{\FigurePath/#1}
%
%  \findmax{\DataPath/outputs.txt}{nr_dof}{\maxdof}
%  \findmin{\DataPath/outputs.txt}{nr_dof}{\mindof}
%  \findmaxDbl{\DataPath/outputs.txt}{Error}{\maxerror}
%  \findminDbl{\DataPath/outputs.txt}{Error}{\minerror}
%
%  \begin{tikzpicture}
%    \begin{axis}[name=mainerrorplot,
%        xlabel={Number of dof},
%        ylabel=Relative error in \%,
%        ymode=log,
%        xmode=log,
%        ymin=1e-7,
%        ymax=1e3,
%        xmin=\mindof,
%        xmax=\maxdof,
%        width=0.9\textwidth,height=0.85\plotwidthmesh,
%        ylabel near ticks,
%        xlabel near ticks,
%        enlargelimits=true,
%        legend style={draw=black,fill=white,legend cell align=left, at={(0.5,1.01)}, anchor=south},
%        legend columns=-1
%      ]
%      \addplot+[ line width=1pt, select coords between index={0}{#2}] table[x expr=\thisrow{nr_dof},y expr=\thisrow{Error}] {\FigurePath/#1/outputs.txt};
%    \end{axis}
%  \end{tikzpicture}
%}



%%%%%%%%%%%%%%%%%%%%%%%%%%%%%%%%%%%%%%%%%%%%%%%%%%%%%%%% OLD



\tikzset{mark size=3}

%:\plothp
\newcommand{\plothp}[2]{

\newcommand{\ResultPath}{Results}
\newcommand{\FigurePath}{Figures}
\renewcommand{\DataPath}{\FigurePath/#1/hp/p2/unref_0.1/unref_h_0.3}
\findmax{\DataPath/outputs.txt}{iter}{\endIter}

\pgfmathparse{subtract(#2,0)} 
\pgfmathtruncatemacro\index{\pgfmathresult}
\getelem{\DataPath/outputs.txt}{nr_dof}{\index}{\nrdof}
\getelemdbl{\DataPath/outputs.txt}{Error}{\index}{\error}

\only<+>{f
\begin{figure}
\centering
\FormatIntegerTwoDigit{#2}{\Iterformated}
\plothpmesh{#1/hp/p2/unref_0.1/unref_h_0.3}{\Iterformated}%
\caption{%
\ifthenelse{\equal{#1}{sin1D}}{Regular}{%
\ifthenelse{\equal{#1}{singular1D}}{Singular}{%
\ifthenelse{\equal{#1}{atan1D}}{Shock}{prout!}%
}%
} Problem. Solution and adapted $hp$-mesh. The drawn mesh needs %
\nrdof~dofs for an error of \pgfmathprintnumber[std, precision=2]{\error}\%.}
\end{figure}
}
\only<+>{
%\quad\quad\quad
\begin{figure}
\centering
%\errorplot{#1}{#3}{#4}
\errorplotlog{#1}{Error}%
\caption{%
\ifthenelse{\equal{#1}{sin1D}}{Regular}{%
\ifthenelse{\equal{#1}{singular1D}}{Singular}{%
\ifthenelse{\equal{#1}{atan1D}}{Shock}{prout!}%
}%
} Problem. Evolution of the error in $H^1$ semi-norm.}
\end{figure}
}
}

\newcommand{\boxplotmagochecksize}[3][]{

%\ResultPath
  \renewcommand{\DataPath}{#1/#2/mat_check_1k}
  \pgfplotstableread[header=false]{\DataPath/QoI_error_boxplot.csv}{\mytable}

 \pgfplotstablegetelem{0}{[index]0}\of{\mytable}
\pgfmathsetmacro\lowwiskeruno{\pgfplotsretval}

\pgfplotstablegetelem{1}{[index]0}\of{\mytable}
\pgfmathsetmacro\lowquartileuno{\pgfplotsretval}

\pgfplotstablegetelem{2}{[index]0}\of{\mytable}
\pgfmathsetmacro\medianuno{\pgfplotsretval}

\pgfplotstablegetelem{3}{[index]0}\of{\mytable}
\pgfmathsetmacro\uppquartileuno{\pgfplotsretval}

\pgfplotstablegetelem{4}{[index]0}\of{\mytable}
\pgfmathsetmacro\uppwiskeruno{\pgfplotsretval}


  \renewcommand{\DataPath}{#1/#2/mat_check_10k}
  \pgfplotstableread[header=false]{\DataPath/QoI_error_boxplot.csv}{\mytable}

 \pgfplotstablegetelem{0}{[index]0}\of{\mytable}
\pgfmathsetmacro\lowwiskerdos{\pgfplotsretval}

\pgfplotstablegetelem{1}{[index]0}\of{\mytable}
\pgfmathsetmacro\lowquartiledos{\pgfplotsretval}

\pgfplotstablegetelem{2}{[index]0}\of{\mytable}
\pgfmathsetmacro\mediandos{\pgfplotsretval}

\pgfplotstablegetelem{3}{[index]0}\of{\mytable}
\pgfmathsetmacro\uppquartiledos{\pgfplotsretval}

\pgfplotstablegetelem{4}{[index]0}\of{\mytable}
\pgfmathsetmacro\uppwiskerdos{\pgfplotsretval}

  \renewcommand{\DataPath}{#1/#2/mat_check_100k}
  \pgfplotstableread[header=false]{\DataPath/QoI_error_boxplot.csv}{\mytable}

 \pgfplotstablegetelem{0}{[index]0}\of{\mytable}
\pgfmathsetmacro\lowwiskertres{\pgfplotsretval}

\pgfplotstablegetelem{1}{[index]0}\of{\mytable}
\pgfmathsetmacro\lowquartiletres{\pgfplotsretval}

\pgfplotstablegetelem{2}{[index]0}\of{\mytable}
\pgfmathsetmacro\mediantres{\pgfplotsretval}

\pgfplotstablegetelem{3}{[index]0}\of{\mytable}
\pgfmathsetmacro\uppquartiletres{\pgfplotsretval}

\pgfplotstablegetelem{4}{[index]0}\of{\mytable}
\pgfmathsetmacro\uppwiskertres{\pgfplotsretval}

\begin{figure}
  \centering
  \begin{tikzpicture}
  \begin{axis}
    [
      boxplot/draw direction=y,
       boxplot/variable width,
      x axis line style={opacity=0},
      axis x line*=bottom,
      axis y line=left,
      enlarge y limits,
      ymajorgrids,
      xtick={1,2,3,4,5},
      xticklabels={1k, 10k, 100k},
      ymode=log,
      height=\plotheight,
      width=0.8\plotwidth,
%      ymin=1e-2,
%      ymax=1e1,
      xlabel={Number of \textbf{validation} samples},
      ylabel={Relative error in the QoI ($\%$)}
    ]
     \addplot+[
      boxplot prepared={
          median=\medianuno,
          upper quartile=\uppquartileuno,
          lower quartile=\lowquartileuno,
          upper whisker=\uppwiskeruno,
          lower whisker=\lowwiskeruno,
%          every box/.style={very thick,dashed,draw=black,fill=yellow},
%         every whisker/.style={red,ultra thick},
%    	 every median/.style={densely dotted,cyan,ultra thick},
        },
    ] coordinates {}; 
    \addplot+[
      boxplot prepared={
          median=\mediandos,
          upper quartile=\uppquartiledos,
          lower quartile=\lowquartiledos,
          upper whisker=\uppwiskerdos,
          lower whisker=\lowwiskerdos,
%          every box/.style={very thick,dashed,draw=black,fill=yellow},
%    	 every whisker/.style={red,ultra thick},
%    	 every median/.style={densely dotted,cyan,ultra thick},
        },
    ] coordinates {};
    \addplot+[
      boxplot prepared={
          median=\mediantres,
          upper quartile=\uppquartiletres,
          lower quartile=\lowquartiletres,
          upper whisker=\uppwiskertres,
          lower whisker=\lowwiskertres
        },
    ] coordinates {};
%%    \addplot+[
%%      boxplot prepared={
%%          median=\mediancin,
%%          upper quartile=\uppquartilecin,
%%          lower quartile=\lowquartilecin,
%%          upper whisker=\uppwiskercin,
%%          lower whisker=\lowwiskercin,
%%          every box/.style={fill=gray!70}
%%%         every whisker/.style={red,ultra thick},
%%%    	 every median/.style={densely dotted,cyan,ultra thick}
%%        },
%%    ] coordinates {};
  \end{axis}
\end{tikzpicture}
 \caption{Different number of testing samples in an \textbf{adaptive grid of #3 samples.}}
\end{figure}
}


%%%%%%%%%%%%%%%%
% \boxplotmagoTimeAdapt
%%%%%%%%%%%%%%%%


\newcommand{\boxplotmagoTimeAdapt}[5][]{

%\ResultPath
  \renewcommand{\DataPath}{#1/#3/mat_check#2}
  \pgfplotstableread[header=false]{\DataPath/QoI_error_boxplot.csv}{\mytable}

 \pgfplotstablegetelem{0}{[index]0}\of{\mytable}
\pgfmathsetmacro\lowwiskeruno{\pgfplotsretval}

\pgfplotstablegetelem{1}{[index]0}\of{\mytable}
\pgfmathsetmacro\lowquartileuno{\pgfplotsretval}

\pgfplotstablegetelem{2}{[index]0}\of{\mytable}
\pgfmathsetmacro\medianuno{\pgfplotsretval}

\pgfplotstablegetelem{3}{[index]0}\of{\mytable}
\pgfmathsetmacro\uppquartileuno{\pgfplotsretval}

\pgfplotstablegetelem{4}{[index]0}\of{\mytable}
\pgfmathsetmacro\uppwiskeruno{\pgfplotsretval}


  \renewcommand{\DataPath}{#1/#4/mat_check#2}
  \pgfplotstableread[header=false]{\DataPath/QoI_error_boxplot.csv}{\mytable}

 \pgfplotstablegetelem{0}{[index]0}\of{\mytable}
\pgfmathsetmacro\lowwiskerdos{\pgfplotsretval}

\pgfplotstablegetelem{1}{[index]0}\of{\mytable}
\pgfmathsetmacro\lowquartiledos{\pgfplotsretval}

\pgfplotstablegetelem{2}{[index]0}\of{\mytable}
\pgfmathsetmacro\mediandos{\pgfplotsretval}

\pgfplotstablegetelem{3}{[index]0}\of{\mytable}
\pgfmathsetmacro\uppquartiledos{\pgfplotsretval}

\pgfplotstablegetelem{4}{[index]0}\of{\mytable}
\pgfmathsetmacro\uppwiskerdos{\pgfplotsretval}

  \renewcommand{\DataPath}{#1/#5/trapi_adapt100_100k}
  \pgfplotstableread[header=false]{\DataPath/QoI_error_boxplot.csv}{\mytable}

 \pgfplotstablegetelem{0}{[index]0}\of{\mytable}
\pgfmathsetmacro\lowwiskertres{\pgfplotsretval}

\pgfplotstablegetelem{1}{[index]0}\of{\mytable}
\pgfmathsetmacro\lowquartiletres{\pgfplotsretval}

\pgfplotstablegetelem{2}{[index]0}\of{\mytable}
\pgfmathsetmacro\mediantres{\pgfplotsretval}

\pgfplotstablegetelem{3}{[index]0}\of{\mytable}
\pgfmathsetmacro\uppquartiletres{\pgfplotsretval}

\pgfplotstablegetelem{4}{[index]0}\of{\mytable}
\pgfmathsetmacro\uppwiskertres{\pgfplotsretval}


\begin{figure}
  \centering
  \begin{tikzpicture}
  \begin{axis}
    [
      boxplot/draw direction=y,
       boxplot/variable width,
      x axis line style={opacity=0},
      axis x line*=bottom,
      axis y line=left,
      enlarge y limits,
      ymajorgrids,
      xtick={1,2,3,4,5},
      xticklabels={1, 10, 100},
      yticklabels={},
      ymode=log,
      height=\plotheight,
      width=0.8\plotwidth,
%      ymin=1e-2,
      ymax=200,
      xlabel={Number of samples for the \textbf{adaptive} process},
%      ylabel={Relative error in the QoI ($\%$)}
    ]
     \addplot+[
      boxplot prepared={
          median=\medianuno,
          upper quartile=\uppquartileuno,
          lower quartile=\lowquartileuno,
          upper whisker=\uppwiskeruno,
          lower whisker=\lowwiskeruno,
%          every box/.style={very thick,dashed,draw=black,fill=yellow},
%         every whisker/.style={red,ultra thick},
%    	 every median/.style={densely dotted,cyan,ultra thick},
        },
    ] coordinates {}; 
    \addplot+[
      boxplot prepared={
          median=\mediandos,
          upper quartile=\uppquartiledos,
          lower quartile=\lowquartiledos,
          upper whisker=\uppwiskerdos,
          lower whisker=\lowwiskerdos,
%          every box/.style={very thick,dashed,draw=black,fill=yellow},
%    	 every whisker/.style={red,ultra thick},
%    	 every median/.style={densely dotted,cyan,ultra thick},
        },
    ] coordinates {};
    \addplot+[
      boxplot prepared={
          median=\mediantres,
          upper quartile=\uppquartiletres,
          lower quartile=\lowquartiletres,
          upper whisker=\uppwiskertres,
          lower whisker=\lowwiskertres
        },
    ] coordinates {};

\node[red] at (1,300) {\large $\sim10$s};
\node[blue] at (2,300) {\large$\sim1$min};
\node[black!60!green!80] at (3,300) {\large $\sim16$h};
%\node[black] at (4,300) {\large 1.5h};
  
   \end{axis}
\end{tikzpicture}
 \caption{\textbf{Execution times} in a Mac laptop for grids with a threshold max-error of 1 $\%$.}
% \caption{Box plot for \textbf{different adaptive grids} with a threshold error of 1 $\%$}
\end{figure}
}

%%%%%%%%%%%%%%%%
% \boxplotmagoTime
%%%%%%%%%%%%%%%%

\newcommand{\boxplotmagoTime}[5][]{

%\ResultPath
  \renewcommand{\DataPath}{#1/#3/mat_check#2}
  \pgfplotstableread[header=false]{\DataPath/QoI_error_boxplot.csv}{\mytable}

 \pgfplotstablegetelem{0}{[index]0}\of{\mytable}
\pgfmathsetmacro\lowwiskeruno{\pgfplotsretval}

\pgfplotstablegetelem{1}{[index]0}\of{\mytable}
\pgfmathsetmacro\lowquartileuno{\pgfplotsretval}

\pgfplotstablegetelem{2}{[index]0}\of{\mytable}
\pgfmathsetmacro\medianuno{\pgfplotsretval}

\pgfplotstablegetelem{3}{[index]0}\of{\mytable}
\pgfmathsetmacro\uppquartileuno{\pgfplotsretval}

\pgfplotstablegetelem{4}{[index]0}\of{\mytable}
\pgfmathsetmacro\uppwiskeruno{\pgfplotsretval}


  \renewcommand{\DataPath}{#1/#4/mat_check#2}
  \pgfplotstableread[header=false]{\DataPath/QoI_error_boxplot.csv}{\mytable}

 \pgfplotstablegetelem{0}{[index]0}\of{\mytable}
\pgfmathsetmacro\lowwiskerdos{\pgfplotsretval}

\pgfplotstablegetelem{1}{[index]0}\of{\mytable}
\pgfmathsetmacro\lowquartiledos{\pgfplotsretval}

\pgfplotstablegetelem{2}{[index]0}\of{\mytable}
\pgfmathsetmacro\mediandos{\pgfplotsretval}

\pgfplotstablegetelem{3}{[index]0}\of{\mytable}
\pgfmathsetmacro\uppquartiledos{\pgfplotsretval}

\pgfplotstablegetelem{4}{[index]0}\of{\mytable}
\pgfmathsetmacro\uppwiskerdos{\pgfplotsretval}

  \renewcommand{\DataPath}{#1/#5/trapi_adapt100_100k}
  \pgfplotstableread[header=false]{\DataPath/QoI_error_boxplot.csv}{\mytable}

 \pgfplotstablegetelem{0}{[index]0}\of{\mytable}
\pgfmathsetmacro\lowwiskertres{\pgfplotsretval}

\pgfplotstablegetelem{1}{[index]0}\of{\mytable}
\pgfmathsetmacro\lowquartiletres{\pgfplotsretval}

\pgfplotstablegetelem{2}{[index]0}\of{\mytable}
\pgfmathsetmacro\mediantres{\pgfplotsretval}

\pgfplotstablegetelem{3}{[index]0}\of{\mytable}
\pgfmathsetmacro\uppquartiletres{\pgfplotsretval}

\pgfplotstablegetelem{4}{[index]0}\of{\mytable}
\pgfmathsetmacro\uppwiskertres{\pgfplotsretval}

\begin{figure}
  \centering
  \begin{tikzpicture}
  \begin{axis}
    [
      boxplot/draw direction=y,
       boxplot/variable width,
      x axis line style={opacity=0},
      axis x line*=bottom,
      axis y line=left,
      enlarge y limits,
      ymajorgrids,
      xtick={1,2,3,4,5},
      xticklabels={1, 10, 100},
      yticklabels={},
      ymode=log,
      height=\plotheight,
      width=0.8\plotwidth,
%      ymin=1e-2,
      ymax=200,
      xlabel={Number of samples for the \textbf{adaptive} process},
%      ylabel={Relative error in the QoI ($\%$)}
    ]
     \addplot+[
      boxplot prepared={
          median=\medianuno,
          upper quartile=\uppquartileuno,
          lower quartile=\lowquartileuno,
          upper whisker=\uppwiskeruno,
          lower whisker=\lowwiskeruno,
        },
    ] coordinates {}; 
    \addplot+[
      boxplot prepared={
          median=\mediandos,
          upper quartile=\uppquartiledos,
          lower quartile=\lowquartiledos,
          upper whisker=\uppwiskerdos,
          lower whisker=\lowwiskerdos,
        },
    ] coordinates {};
    \addplot+[
      boxplot prepared={
          median=\mediantres,
          upper quartile=\uppquartiletres,
          lower quartile=\lowquartiletres,
          upper whisker=\uppwiskertres,
          lower whisker=\lowwiskertres
        },
    ] coordinates {};

\node[red] at (1,300) {\large $\sim10$min};
\node[blue] at (2,300) {\large$\sim80$min};
\node[black!60!green!80] at (3.1,300) {\large $\sim28$h (1sam/s)};
%\node[black] at (4,300) {\large 1.5h};
  
   \end{axis}
\end{tikzpicture}
 \caption{\textbf{Execution times} in a Mac laptop for grids with a threshold max-error of 1 $\%$.}
% \caption{Box plot for \textbf{different adaptive grids} with a threshold error of 1 $\%$}
\end{figure}
}

%%%%%%%%%%%%%%%%
% \boxplotmagoNdof
%%%%%%%%%%%%%%%%

\newcommand{\boxplotmagoNdof}[6][]{

%\ResultPath
  \renewcommand{\DataPath}{#1/mat_check_#3}
  \pgfplotstableread[header=false]{\DataPath/QoI_error_boxplot.csv}{\mytable}

 \pgfplotstablegetelem{0}{[index]0}\of{\mytable}
\pgfmathsetmacro\lowwiskeruno{\pgfplotsretval}

\pgfplotstablegetelem{1}{[index]0}\of{\mytable}
\pgfmathsetmacro\lowquartileuno{\pgfplotsretval}

\pgfplotstablegetelem{2}{[index]0}\of{\mytable}
\pgfmathsetmacro\medianuno{\pgfplotsretval}

\pgfplotstablegetelem{3}{[index]0}\of{\mytable}
\pgfmathsetmacro\uppquartileuno{\pgfplotsretval}

\pgfplotstablegetelem{4}{[index]0}\of{\mytable}
\pgfmathsetmacro\uppwiskeruno{\pgfplotsretval}


  \renewcommand{\DataPath}{#1/mat_check_#4}
  \pgfplotstableread[header=false]{\DataPath/QoI_error_boxplot.csv}{\mytable}

 \pgfplotstablegetelem{0}{[index]0}\of{\mytable}
\pgfmathsetmacro\lowwiskerdos{\pgfplotsretval}

\pgfplotstablegetelem{1}{[index]0}\of{\mytable}
\pgfmathsetmacro\lowquartiledos{\pgfplotsretval}

\pgfplotstablegetelem{2}{[index]0}\of{\mytable}
\pgfmathsetmacro\mediandos{\pgfplotsretval}

\pgfplotstablegetelem{3}{[index]0}\of{\mytable}
\pgfmathsetmacro\uppquartiledos{\pgfplotsretval}

\pgfplotstablegetelem{4}{[index]0}\of{\mytable}
\pgfmathsetmacro\uppwiskerdos{\pgfplotsretval}

  \renewcommand{\DataPath}{#1/mat_check_#5}
  \pgfplotstableread[header=false]{\DataPath/QoI_error_boxplot.csv}{\mytable}

 \pgfplotstablegetelem{0}{[index]0}\of{\mytable}
\pgfmathsetmacro\lowwiskertres{\pgfplotsretval}

\pgfplotstablegetelem{1}{[index]0}\of{\mytable}
\pgfmathsetmacro\lowquartiletres{\pgfplotsretval}

\pgfplotstablegetelem{2}{[index]0}\of{\mytable}
\pgfmathsetmacro\mediantres{\pgfplotsretval}

\pgfplotstablegetelem{3}{[index]0}\of{\mytable}
\pgfmathsetmacro\uppquartiletres{\pgfplotsretval}

\pgfplotstablegetelem{4}{[index]0}\of{\mytable}
\pgfmathsetmacro\uppwiskertres{\pgfplotsretval}

  \renewcommand{\DataPath}{#1/mat_check_#6}
  \pgfplotstableread[header=false]{\DataPath/QoI_error_boxplot.csv}{\mytable}

 \pgfplotstablegetelem{0}{[index]0}\of{\mytable}
\pgfmathsetmacro\lowwiskercua{\pgfplotsretval}

\pgfplotstablegetelem{1}{[index]0}\of{\mytable}
\pgfmathsetmacro\lowquartilecua{\pgfplotsretval}

\pgfplotstablegetelem{2}{[index]0}\of{\mytable}
\pgfmathsetmacro\mediancua{\pgfplotsretval}

\pgfplotstablegetelem{3}{[index]0}\of{\mytable}
\pgfmathsetmacro\uppquartilecua{\pgfplotsretval}

\pgfplotstablegetelem{4}{[index]0}\of{\mytable}
\pgfmathsetmacro\uppwiskercua{\pgfplotsretval}

\begin{figure}
  \centering
  \begin{tikzpicture}
  \begin{axis}
    [
      boxplot/draw direction=y,
       boxplot/variable width,
      x axis line style={opacity=0},
      axis x line*=bottom,
      axis y line=left,
      enlarge y limits,
      ymajorgrids,
      xtick={1,2,3,4},
      xticklabels={1, 5, 10, 100},
      ymode=log,
      height=\plotheight,
      width=0.8\plotwidth,
%      ymin=1e-2,
%      ymax=1e1,
      xlabel={Number of samples for the \textbf{adaptive} process},
      ylabel={Relative error in the QoI ($\%$)}
    ]
     \addplot+[
      boxplot prepared={
          median=\medianuno,
          upper quartile=\uppquartileuno,
          lower quartile=\lowquartileuno,
          upper whisker=\uppwiskeruno,
          lower whisker=\lowwiskeruno,
%          every box/.style={very thick,dashed,draw=black,fill=yellow},
%         every whisker/.style={red,ultra thick},
%    	 every median/.style={densely dotted,cyan,ultra thick},
        },
    ] coordinates {}; 
    \addplot+[
      boxplot prepared={
          median=\mediandos,
          upper quartile=\uppquartiledos,
          lower quartile=\lowquartiledos,
          upper whisker=\uppwiskerdos,
          lower whisker=\lowwiskerdos,
%          every box/.style={very thick,dashed,draw=black,fill=yellow},
%    	 every whisker/.style={red,ultra thick},
%    	 every median/.style={densely dotted,cyan,ultra thick},
        },
    ] coordinates {};
    \addplot+[
      boxplot prepared={
          median=\mediantres,
          upper quartile=\uppquartiletres,
          lower quartile=\lowquartiletres,
          upper whisker=\uppwiskertres,
          lower whisker=\lowwiskertres
        },
    ] coordinates {};
    \addplot+[
      boxplot prepared={
          median=\mediancua,
          upper quartile=\uppquartilecua,
          lower quartile=\lowquartilecua,
          upper whisker=\uppwiskercua,
          lower whisker=\lowwiskercua
        },
    ] coordinates {};
  \end{axis}
\end{tikzpicture}
 \caption{Different number of adaptive grids with a \textbf{stopping criteria of #2 DoFs}.}
\end{figure}
}

\newcommand{\boxplotmagocheck}[7][]{

%\ResultPath
  \renewcommand{\DataPath}{#1/#3/mat_check#2}
  \pgfplotstableread[header=false]{\DataPath/QoI_error_boxplot.csv}{\mytable}

 \pgfplotstablegetelem{0}{[index]0}\of{\mytable}
\pgfmathsetmacro\lowwiskeruno{\pgfplotsretval}

\pgfplotstablegetelem{1}{[index]0}\of{\mytable}
\pgfmathsetmacro\lowquartileuno{\pgfplotsretval}

\pgfplotstablegetelem{2}{[index]0}\of{\mytable}
\pgfmathsetmacro\medianuno{\pgfplotsretval}

\pgfplotstablegetelem{3}{[index]0}\of{\mytable}
\pgfmathsetmacro\uppquartileuno{\pgfplotsretval}

\pgfplotstablegetelem{4}{[index]0}\of{\mytable}
\pgfmathsetmacro\uppwiskeruno{\pgfplotsretval}


  \renewcommand{\DataPath}{#1/#4/mat_check#2}
  \pgfplotstableread[header=false]{\DataPath/QoI_error_boxplot.csv}{\mytable}

 \pgfplotstablegetelem{0}{[index]0}\of{\mytable}
\pgfmathsetmacro\lowwiskerdos{\pgfplotsretval}

\pgfplotstablegetelem{1}{[index]0}\of{\mytable}
\pgfmathsetmacro\lowquartiledos{\pgfplotsretval}

\pgfplotstablegetelem{2}{[index]0}\of{\mytable}
\pgfmathsetmacro\mediandos{\pgfplotsretval}

\pgfplotstablegetelem{3}{[index]0}\of{\mytable}
\pgfmathsetmacro\uppquartiledos{\pgfplotsretval}

\pgfplotstablegetelem{4}{[index]0}\of{\mytable}
\pgfmathsetmacro\uppwiskerdos{\pgfplotsretval}

  \renewcommand{\DataPath}{#1/#5/mat_check#2}
  \pgfplotstableread[header=false]{\DataPath/QoI_error_boxplot.csv}{\mytable}

 \pgfplotstablegetelem{0}{[index]0}\of{\mytable}
\pgfmathsetmacro\lowwiskertres{\pgfplotsretval}

\pgfplotstablegetelem{1}{[index]0}\of{\mytable}
\pgfmathsetmacro\lowquartiletres{\pgfplotsretval}

\pgfplotstablegetelem{2}{[index]0}\of{\mytable}
\pgfmathsetmacro\mediantres{\pgfplotsretval}

\pgfplotstablegetelem{3}{[index]0}\of{\mytable}
\pgfmathsetmacro\uppquartiletres{\pgfplotsretval}

\pgfplotstablegetelem{4}{[index]0}\of{\mytable}
\pgfmathsetmacro\uppwiskertres{\pgfplotsretval}

  \renewcommand{\DataPath}{#1/#6/mat_check#2}
  \pgfplotstableread[header=false]{\DataPath/QoI_error_boxplot.csv}{\mytable}

 \pgfplotstablegetelem{0}{[index]0}\of{\mytable}
\pgfmathsetmacro\lowwiskercua{\pgfplotsretval}

\pgfplotstablegetelem{1}{[index]0}\of{\mytable}
\pgfmathsetmacro\lowquartilecua{\pgfplotsretval}

\pgfplotstablegetelem{2}{[index]0}\of{\mytable}
\pgfmathsetmacro\mediancua{\pgfplotsretval}

\pgfplotstablegetelem{3}{[index]0}\of{\mytable}
\pgfmathsetmacro\uppquartilecua{\pgfplotsretval}

\pgfplotstablegetelem{4}{[index]0}\of{\mytable}
\pgfmathsetmacro\uppwiskercua{\pgfplotsretval}

  \renewcommand{\DataPath}{#1/#7/mat_check#2}
  \pgfplotstableread[header=false]{\DataPath/QoI_error_boxplot.csv}{\mytable}

 \pgfplotstablegetelem{0}{[index]0}\of{\mytable}
\pgfmathsetmacro\lowwiskercin{\pgfplotsretval}

\pgfplotstablegetelem{1}{[index]0}\of{\mytable}
\pgfmathsetmacro\lowquartilecin{\pgfplotsretval}

\pgfplotstablegetelem{2}{[index]0}\of{\mytable}
\pgfmathsetmacro\mediancin{\pgfplotsretval}

\pgfplotstablegetelem{3}{[index]0}\of{\mytable}
\pgfmathsetmacro\uppquartilecin{\pgfplotsretval}

\pgfplotstablegetelem{4}{[index]0}\of{\mytable}
\pgfmathsetmacro\uppwiskercin{\pgfplotsretval}


\begin{figure}
  \centering
  \begin{tikzpicture}
  \begin{axis}
    [
      boxplot/draw direction=y,
       boxplot/variable width,
      x axis line style={opacity=0},
      axis x line*=bottom,
      axis y line=left,
      enlarge y limits,
      ymajorgrids,
      xtick={1,2,3,4,5},
      xticklabels={1, 5, 10, 50, 100},
      ymode=log,
      height=\plotheight,
      width=0.8\plotwidth,
%      ymin=1e-2,
%      ymax=1e1,
      xlabel={Number of samples for the \textbf{adaptive} process},
      ylabel={Relative error in the QoI ($\%$)}
    ]
     \addplot+[
      boxplot prepared={
          median=\medianuno,
          upper quartile=\uppquartileuno,
          lower quartile=\lowquartileuno,
          upper whisker=\uppwiskeruno,
          lower whisker=\lowwiskeruno,
%          every box/.style={very thick,dashed,draw=black,fill=yellow},
%         every whisker/.style={red,ultra thick},
%    	 every median/.style={densely dotted,cyan,ultra thick},
        },
    ] coordinates {}; 
    \addplot+[
      boxplot prepared={
          median=\mediandos,
          upper quartile=\uppquartiledos,
          lower quartile=\lowquartiledos,
          upper whisker=\uppwiskerdos,
          lower whisker=\lowwiskerdos,
%          every box/.style={very thick,dashed,draw=black,fill=yellow},
%    	 every whisker/.style={red,ultra thick},
%    	 every median/.style={densely dotted,cyan,ultra thick},
        },
    ] coordinates {};
    \addplot+[
      boxplot prepared={
          median=\mediantres,
          upper quartile=\uppquartiletres,
          lower quartile=\lowquartiletres,
          upper whisker=\uppwiskertres,
          lower whisker=\lowwiskertres
        },
    ] coordinates {};
    \addplot+[
      boxplot prepared={
          median=\mediancua,
          upper quartile=\uppquartilecua,
          lower quartile=\lowquartilecua,
          upper whisker=\uppwiskercua,
          lower whisker=\lowwiskercua
        },
    ] coordinates {};
    \addplot+[
      boxplot prepared={
          median=\mediancin,
          upper quartile=\uppquartilecin,
          lower quartile=\lowquartilecin,
          upper whisker=\uppwiskercin,
          lower whisker=\lowwiskercin,
%          every box/.style={fill=gray!70}
%         every whisker/.style={red,ultra thick},
%    	 every median/.style={densely dotted,cyan,ultra thick}
        },
    ] coordinates {};
  \end{axis}
\end{tikzpicture}
 \caption{Different number of adaptive grids with a \textbf{threshold max-error of 1 $\%$}.}
% \caption{Box plot for \textbf{different adaptive grids} with a threshold error of 1 $\%$}
\end{figure}
}



\newcommand{\boxplotmago}[5][]{

%\ResultPath
  \renewcommand{\DataPath}{#1/#2/Check/mat_check}
  \pgfplotstableread[header=false]{\DataPath/QoI_error_boxplot.csv}{\mytable}

 \pgfplotstablegetelem{0}{[index]0}\of{\mytable}
\pgfmathsetmacro\lowwiskeruno{\pgfplotsretval}

\pgfplotstablegetelem{1}{[index]0}\of{\mytable}
\pgfmathsetmacro\lowquartileuno{\pgfplotsretval}

\pgfplotstablegetelem{2}{[index]0}\of{\mytable}
\pgfmathsetmacro\medianuno{\pgfplotsretval}

\pgfplotstablegetelem{3}{[index]0}\of{\mytable}
\pgfmathsetmacro\uppquartileuno{\pgfplotsretval}

\pgfplotstablegetelem{4}{[index]0}\of{\mytable}
\pgfmathsetmacro\uppwiskeruno{\pgfplotsretval}


  \renewcommand{\DataPath}{#1/#3/Check/mat_check}
  \pgfplotstableread[header=false]{\DataPath/QoI_error_boxplot.csv}{\mytable}

 \pgfplotstablegetelem{0}{[index]0}\of{\mytable}
\pgfmathsetmacro\lowwiskerdos{\pgfplotsretval}

\pgfplotstablegetelem{1}{[index]0}\of{\mytable}
\pgfmathsetmacro\lowquartiledos{\pgfplotsretval}

\pgfplotstablegetelem{2}{[index]0}\of{\mytable}
\pgfmathsetmacro\mediandos{\pgfplotsretval}

\pgfplotstablegetelem{3}{[index]0}\of{\mytable}
\pgfmathsetmacro\uppquartiledos{\pgfplotsretval}

\pgfplotstablegetelem{4}{[index]0}\of{\mytable}
\pgfmathsetmacro\uppwiskerdos{\pgfplotsretval}


 \renewcommand{\DataPath}{#1/#4/Check/mat_check}
  \pgfplotstableread[header=false]{\DataPath/QoI_error_boxplot.csv}{\mytable}

 \pgfplotstablegetelem{0}{[index]0}\of{\mytable}
\pgfmathsetmacro\lowwiskertres{\pgfplotsretval}

\pgfplotstablegetelem{1}{[index]0}\of{\mytable}
\pgfmathsetmacro\lowquartiletres{\pgfplotsretval}

\pgfplotstablegetelem{2}{[index]0}\of{\mytable}
\pgfmathsetmacro\mediantres{\pgfplotsretval}

\pgfplotstablegetelem{3}{[index]0}\of{\mytable}
\pgfmathsetmacro\uppquartiletres{\pgfplotsretval}

\pgfplotstablegetelem{4}{[index]0}\of{\mytable}
\pgfmathsetmacro\uppwiskertres{\pgfplotsretval}

 \renewcommand{\DataPath}{#1/#5/Check/mat_check}
  \pgfplotstableread[header=false]{\DataPath/QoI_error_boxplot.csv}{\mytable}

 \pgfplotstablegetelem{0}{[index]0}\of{\mytable}
\pgfmathsetmacro\lowwiskercua{\pgfplotsretval}

\pgfplotstablegetelem{1}{[index]0}\of{\mytable}
\pgfmathsetmacro\lowquartilecua{\pgfplotsretval}

\pgfplotstablegetelem{2}{[index]0}\of{\mytable}
\pgfmathsetmacro\mediancua{\pgfplotsretval}

\pgfplotstablegetelem{3}{[index]0}\of{\mytable}
\pgfmathsetmacro\uppquartilecua{\pgfplotsretval}

\pgfplotstablegetelem{4}{[index]0}\of{\mytable}
\pgfmathsetmacro\uppwiskercua{\pgfplotsretval}

\begin{figure}
  \centering
  \begin{tikzpicture}
  \begin{axis}
    [
      boxplot/draw direction=y,
       boxplot/variable width,
      x axis line style={opacity=0},
      axis x line*=bottom,
      axis y line=left,
      enlarge y limits,
      ymajorgrids,
      xtick={1,2,3,4,5},
      xticklabels={1, 5, 10, 100, 100(10k)},
      ymode=log,
      height=\plotheight,
      width=0.8\plotwidth,
%      ymin=1e-2,
%      ymax=1e1,
      xlabel={Number of samples for the \textbf{adaptive} process},
      ylabel={Relative error in the QoI ($\%$)}
    ]
     \addplot+[
      boxplot prepared={
          median=\medianuno,
          upper quartile=\uppquartileuno,
          lower quartile=\lowquartileuno,
          upper whisker=\uppwiskeruno,
          lower whisker=\lowwiskeruno,
%          every box/.style={very thick,dashed,draw=black,fill=yellow},
%         every whisker/.style={red,ultra thick},
%    	 every median/.style={densely dotted,cyan,ultra thick},
        },
    ] coordinates {}; 
    \addplot+[
      boxplot prepared={
          median=\mediandos,
          upper quartile=\uppquartiledos,
          lower quartile=\lowquartiledos,
          upper whisker=\uppwiskerdos,
          lower whisker=\lowwiskerdos,
%          every box/.style={very thick,dashed,draw=black,fill=yellow},
%    	 every whisker/.style={red,ultra thick},
%    	 every median/.style={densely dotted,cyan,ultra thick},
        },
    ] coordinates {};
    \addplot+[
      boxplot prepared={
          median=\mediantres,
          upper quartile=\uppquartiletres,
          lower quartile=\lowquartiletres,
          upper whisker=\uppwiskertres,
          lower whisker=\lowwiskertres
        },
    ] coordinates {};
    \addplot+[
      boxplot prepared={
          median=\mediancua,
          upper quartile=\uppquartilecua,
          lower quartile=\lowquartilecua,
          upper whisker=\uppwiskercua,
          lower whisker=\lowwiskercua
        },
    ] coordinates {};
%    \addplot+[
%      boxplot prepared={
%          median=\mediancin,
%          upper quartile=\uppquartilecin,
%          lower quartile=\lowquartilecin,
%          upper whisker=\uppwiskercin,
%          lower whisker=\lowwiskercin,
%          every box/.style={fill=gray!70}
%%         every whisker/.style={red,ultra thick},
%%    	 every median/.style={densely dotted,cyan,ultra thick}
%        },
%    ] coordinates {};
  \end{axis}
\end{tikzpicture}
 \caption{Different number of adaptive grids with a \textbf{threshold max-error of 0.1 $\%$}.}
\end{figure}
}

%:\plothpunrefMAGO
\newcommand{\plothpunrefmago}[1]{
\renewcommand{\DataPath}{\FigurePath/Unref/#1/hp/order_1}
\findmax{\DataPath/outputs.txt}{iter}{\endIter}

\findmax{\FigurePath/#1/hp/outputs.txt}{iter}{\endIterh}

\FormatIntegerTwoDigit{\endIterh}{\endIterhformated}

\begin{center}

\begin{figure}
\foreach \i in {1,2,...,\endIter}{%
\only<+>{%
\begin{subfigure}{\subplotwidthhp}
\centering
%\tikzset{/tikz/external/export next=false}
\FormatIntegerThreeDigit{\i}{\iformat}
\plothpmagomesh{Unref/#1/hp/order_1}{\iformat}{#1/hp/coarse_mesh_fine_\endIterhformated_Direct_solution_abs}%
%\pgfmathparse{add(\i,0)} 
%\pgfmathtruncatemacro\plotindex{\pgfmathresult}
%\pgfmathparse{add(\endIter,0)} 
%\pgfmathtruncatemacro\lastindex{\pgfmathresult}
%\vspace{10mm}
\subcaption{Adaptive iteration \i ~of \endIter}
\end{subfigure}
\begin{subfigure}{\subplotwidthhp}
\centering
% % %
\errorplotlogUnrefBorrar{Unref/#1/hp/order_1}{\i}
\subcaption{ Evolution of the relative error}
\end{subfigure}
\only<1->{\caption{Adaptive process for a Helmholtz problem}}
}%
}
%}
\end{figure}
\end{center}
}


%:\plothpmagomesh
\newcommand{\plothpmagomesh}[3]{%
\renewcommand{\DataPath}{\FigurePath/#1}
%
%\lastelement{\DataPath/outputs.txt}{MaxP}{\maxp}
%\lastelement{\DataPath/outputs.txt}{MinP}{\minp}
%
%\pgfplotsset{colormap/YlOrRd}
%\tikzset{/tikz/external/export next=false}
\begin{tikzpicture}
\begin{axis}[name=master,
width=\textwidth,height=\textwidth,
enlargelimits=false,
xlabel=$x$,
ylabel=$y$,
xlabel near ticks,
ylabel near ticks,
hide axis,
]
%

\IfSubStr{#1}{square}{%
 \addplot [forget plot] graphics[xmin=-1,xmax=1,ymin=-1,ymax=1] {\DataPath/mesh_X_#2.png};
}{
 \addplot [forget plot] graphics[xmin=0,xmax=1,ymin=0,ymax=1] {\DataPath/mesh_X_#2.png};
 }
 \addplot [forget plot, opacity=0.2] graphics[xmin=0,xmax=1,ymin=0,ymax=1] {\FigurePath/#3};


\end{axis}
\end{tikzpicture}%
}



\newcommand{\plothpmago}[2][]{

  \renewcommand{\DataPath}{\FigurePath/#2}%\PbName/hp
  \findmax{\DataPath/outputs.txt}{iter}{\endIter}

  \FormatIntegerThreeDigit{\endIter}{\endIterformated}

  \foreach \idim in {X,Y}{
      \begin{subfigure}{\subplotwidth} %subplot (a) (b)
        \plothpmeshnew{#2}{\endIterformated}{\idim}
        \caption{$hp$-adapted mesh. $p$ in the \ifthenelse{\equal{\idim}{X}}{$x$}{$y$}-direction}
        \label{fig:mesh_\idim_#2}
      \end{subfigure}
    }
}

%%%%%%%%%%%%%%%%%%%%%%\PLOT_HP_MESH %%%%%%%%%%%%%%%%%%%%%%%%%%%%%%
%%%%%%%%%%%%%%%%%%%%%%: \plothpmesh{#2/hp}{\endIterformated}{\idim} \idim in {X,Y}
\newcommand{\plothpmeshnew}[3]{
  \renewcommand{\DataPath}{\FigurePath/#1}

  \lastelement{\DataPath/outputs.txt}{MaxP}{\maxp}
  \lastelement{\DataPath/outputs.txt}{MinP}{\minp}

  %\pgfplotsset{colormap/jet} %Barra de color
  \pgfplotsset{colormap/YlOrRd} %Barra de color
  %\tikzset{/tikz/external/export next=false}
  \begin{tikzpicture}
    \begin{axis}[name=master,
        width=\textwidth,height=\textwidth,
        enlargelimits=false,
        xlabel=$x$,
        ylabel=$y$,
        xlabel near ticks,
        ylabel near ticks,
        colorbar horizontal,
        colorbar style={%
            xtick={1,2,...,\maxp},
            title={Order of approximation},
            at={(master.above north west)},anchor=south west,
            yshift=2pt,
            title style={
                yshift=2pt
              },
            yshift=0.2em,
            xticklabel pos=upper,
          },
        %point meta min=\minp,
        point meta min=1,
        point meta max=\maxp,
      ]
      %
      \addplot [forget plot] graphics[xmin=0,xmax=1,ymin=0,ymax=1] {\DataPath/mesh_#3_#2.png}; %Li­mites del dominio de los subplots (a) y (b). #3 X o Y, #2 por ejemplo 015

    \end{axis}
  \end{tikzpicture}
}


\newcommand{\plothpmesh}[2]{
\pgfplotsset{colormap/YlOrRd}
\renewcommand{\DataPath}{\FigurePath/#1}

%\getelem{\DataPath/outputs.txt}{iter}{#2}{\iter}
\findmax{\DataPath/coarse_mesh_#2.txt}{order}{\maxp}
%\findmin{\DataPath/initial_mesh.txt}{order}{\minp}

\begin{tikzpicture}
\begin{axis}[name=master,
width=\plotwidth,height=0.8\plotheight,
enlargelimits=false,
enlarge y limits=true,
enlarge y limits=0.2,
%point meta min=\minp,
point meta min=1,
point meta max=\maxp,
xlabel=$x$,
ylabel=$u(x)$ ,
]

%\addplot[line width=1pt, opacity=0.2] table[x=X, y=My_beautiful_solution_real] {\DataPath/sol#2.txt};

\addplot[solid, line width=1pt, color=black] table[x=X, y=My_beautiful_exact_solution_real] {\DataPath/exact_sol.txt};
\end{axis}

\begin{axis}[
%colormap access=direct,
%colormap access=piecewise const,
%colorbar sampled,
%colorbar right,
colorbar horizontal,
  colorbar style={ %samples=5,
%    ytick=data,
%    y tick label as interval=true,
%	at={(master.east)}, anchor=west,
	xtick={1,2,...,\maxp},
         at={(master.above north west)},anchor=south west,
    	title=Approximation order $p$,
	xticklabel pos=upper,
	yshift=1em,
	title style={yshift=2pt},
	%	height=\plotheight/1.5%\pgfkeysvalueof{/pgfplots/master axis height}
} ,
width=\plotwidth,%
%at={(master.above north west)},anchor=south west,
at={(master.south west)},%anchor=south west,
enlargelimits=false,
%enlarge x limits=true,
axis lines=none,
height=2cm,%\pgfkeysvalueof{/pgfplots/colorbar/width},
ticks=none,
%point meta min=\minp,
point meta min=1,
point meta max=\maxp,
title=Mesh,
name=coarse_mesh,
title style={left,at={(0,0)}, align=right},
 ]
    \addplot [patch, patch type=rectangle, point meta=explicit,line width=0.6pt, faceted color= black]  table[meta=order] {\DataPath/coarse_mesh_#2.txt};%\mesh};

\end{axis}

\IfSubStr{#1}{singular1D}{%
\begin{axis}[
%
width=0.8\textwidth,%
at={($(coarse_mesh.north)+(1em,1em)$)},anchor=south ,
enlargelimits=false,
%%enlarge x limits=true,
%axis lines=none,
height=2cm,%\pgfkeysvalueof{/pgfplots/colorbar/width},
ticks=none,
xmin=0,
xmax=1.4901161193847656e-008,
%ytick style={draw=none},
%yticklabels={},
%xtick={0,5e-9,1e-8},
%xticklabels={0, $5\cdot10^{-9}$, $10^{-8}$},
%%point meta min=\minp,
point meta min=1,
point meta max=\maxp,
title=Zoomed Mesh ($\times 10^{8}$),
name=coarse_mesh_zoom,
title style={above, yshift=-0.5em},
 ]
    \addplot [patch, patch type=rectangle, point meta=explicit,line width=0.6pt, faceted color= black]  table[meta=order] {\DataPath/coarse_mesh_#2_zoom.txt};
%
\end{axis}

\draw [<-, thick, dashed] (coarse_mesh_zoom.west) to [out=180, in=90] (coarse_mesh.north west);

}
{}

\end{tikzpicture}

}


%:\errorplotlog
\newcommand{\errorplotlog}[2]{

%\tikzset{/tikz/external/export next=false}
\begin{tikzpicture}
\pgfplotsset{xmode=log}

\begin{axis}[name=mainerrorplot,
xlabel={Number of DoFs},
%x dir=reverse,
ylabel=Relative error in \% (log scale),
ymode=log,
xmode=log,
%yticklabel pos=right,
width=\plotwidth,height=\plotheight,
ylabel near ticks,
xlabel near ticks,
enlargelimits=true,
legend style={draw=black,fill=white,legend cell align=left, at={(0.5,1.01)}, anchor=south},
legend columns=-1
]

\addplot+[ line width=1pt] table[x expr=\thisrow{nr_dof},y expr=\thisrow{#2}] {\FigurePath/#1/hp/p2/unref_0.1/unref_h_0.3/outputs.txt}node[pos=0.9, pin={[pin edge=solid]180:$hp$}]{};
%\addlegendentry{$hp$}

%\IfSubStr{#1}{lap2DLshape}{%
%\addplot+[ line width=1pt] table[x expr=\thisrow{nr_dof},y expr=\thisrow{#2}] {\FigurePath/#1/h/unref_0.1/unref_h_0.3/order_1/outputs.txt}node[pos=1, pin={[pin edge=solid]90:$h$ ($p=1$)}]{};
%}{%
%\IfSubStr{#1}{maxwellGOA}{%
%\addplot+[ line width=1pt] table[x expr=\thisrow{nr_dof},y expr=\thisrow{#2}] {\FigurePath/#1/h/unref_0.1/unref_h_0.3/order_4/outputs.txt}node[pos=1, pin={[pin edge=solid, anchor=north]-60:$h$ ($p=4$)}]{};
%}{%
%\IfSubStr{#1}{DiffConv2D}{%
%\addplot+[ line width=1pt] table[x expr=\thisrow{nr_dof},y expr=\thisrow{#2}] {\FigurePath/#1/h/unref_0.1/unref_h_0.3/order_1/outputs.txt}node[pos=0.95, pin={[pin edge=solid]90:$h$ ($p=1$)}]{};
%}}}

\IfSubStr{#1}{singular2D}{%
\addplot+[ line width=1pt] table[x expr=\thisrow{nr_dof},y expr=\thisrow{#2}] {\FigurePath/#1/h/unref_0.1/unref_h_0.3/order_2/outputs.txt} node[pos=0.95, pin={[pin edge=solid, pin distance=0.2cm]90:$h$ ($p=2$)}]{};%
}{%
\IfSubStr{#1}{atancircle2D}{%
\addplot+[ line width=1pt] table[x expr=\thisrow{nr_dof},y expr=\thisrow{#2}] {\FigurePath/#1/h/unref_0.1/unref_h_0.3/order_2/outputs.txt} node[pos=0.98, pin={[pin edge=solid]-90:$h$ ($p=2$)}]{};%
}{%
%\IfSubStr{#1}{lap2DLshape}{%
%\addplot+[ line width=1pt] table[x expr=\thisrow{nr_dof},y expr=\thisrow{#2}] {\FigurePath/#1/h/unref_0.1/unref_h_0.3/order_2/outputs.txt} node[pos=0.1, pin={[pin edge=solid]0:$h$ ($p=2$)}]{};%
%}{%
%\IfSubStr{#1}{Helm2DGOA}{%
%\addplot+[ line width=1pt] table[x expr=\thisrow{nr_dof},y expr=\thisrow{#2}] {\FigurePath/#1/h/unref_0.1/unref_h_0.3/order_2/outputs.txt} node[pos=0.95, pin={[pin edge=solid]180:$h$ ($p=2$)}]{};%
%}{%
\IfSubStr{#1}{DiffConv2D}{%
\addplot+[ line width=1pt] table[x expr=\thisrow{nr_dof},y expr=\thisrow{#2}] {\FigurePath/#1/h/unref_0.1/unref_h_0.3/order_2/outputs.txt} node[pos=0.1, pin={[pin edge=solid]0:$h$ ($p=2$)}]{};%
}{%
\addplot+[ line width=1pt] table[x expr=\thisrow{nr_dof},y expr=\thisrow{#2}] {\FigurePath/#1/h/unref_0.1/unref_h_0.3/order_2/outputs.txt} node[pos=0.9, pin={[pin edge=solid]-90:$h$ ($p=2$)}]{};%
}%
%}
%}
}
}



%\addlegendentry{$h$ ($p=2$)}

%\addplot+[ line width=1pt] table[x expr=\thisrow{nr_dof},y expr=\thisrow{#2}] {\FigurePath/#1/p/unref_0.2/outputs.txt};
%\addlegendentry{$p$}

%\begin{scope}[yshift=0.5em]
%\ifthenelse{#2 = 1}{
%\addplot[color=red, thin] [domain=\maxdof/4:\maxdof, samples=10]{exp(-#3*x^(1/#2))*\minerror/exp(-#3*\maxdof^(1/#2))} node [pos=0.3](pin){};
%%,pin={60:$e^{-#3N}$}]{};
%}%
%{
%\addplot[color=red, thin] [domain=1:\maxdof, samples=10]{exp(-#3*x^(1/#2))*\minerror/exp(-#3*\maxdof^(1/#2))} node [pos=0.3](pin){};
%}
%
%\end{scope}



\end{axis}
\end{tikzpicture}
}

%:\errorplotsemilog
\newcommand{\errorplotsemilog}[3]{

%\tikzset{/tikz/external/export next=false}
\begin{tikzpicture}
\pgfplotsset{xmode=log}

\begin{axis}[name=mainerrorplot,
xlabel={Number of dof$^{#3}$},
%x dir=reverse,
ylabel=Relative error in \%,
ymode=log,
%xmode=log,
%yticklabel pos=right,
width=\plotwidth,height=\plotheight,
ylabel near ticks,
xlabel near ticks,
enlargelimits=true,
legend style={draw=black,fill=white,legend cell align=left, at={(0.5,1.01)}, anchor=south},
legend columns=-1
]

\addplot+[ line width=1pt] table[x expr=\thisrow{nr_dof}^(#3),y expr=\thisrow{#2}] {\FigurePath/#1/hp/p2/unref_0.1/unref_h_0.3/outputs.txt}node[pos=0.9, pin={[pin edge=solid]180:$hp$}]{};
%\addlegendentry{$hp$}

\IfSubStr{#1}{lap2DLshape}{%
\addplot+[ line width=1pt] table[x expr=\thisrow{nr_dof}^(#3),y expr=\thisrow{#2}] {\FigurePath/#1/h/unref_0.1/unref_h_0.3/order_1/outputs.txt}node[pos=1, pin={[pin edge=solid]90:$h$ ($p=1$)}]{};
}{%
\IfSubStr{#1}{maxwellGOA}{%
\addplot+[ line width=1pt] table[x expr=\thisrow{nr_dof}^(#3),y expr=\thisrow{#2}] {\FigurePath/#1/h/unref_0.1/unref_h_0.3/order_4/outputs.txt}node[pos=1, pin={[pin edge=solid, anchor=north]-60:$h$ ($p=4$)}]{};
}{%
\IfSubStr{#1}{DiffConv2D}{%
\addplot+[ line width=1pt] table[x expr=\thisrow{nr_dof}^(#3),y expr=\thisrow{#2}] {\FigurePath/#1/h/unref_0.1/unref_h_0.3/order_1/outputs.txt}node[pos=0.95, pin={[pin edge=solid]90:$h$ ($p=1$)}]{};
}{%
\addplot+[ line width=1pt] table[x expr=\thisrow{nr_dof}^(#3),y expr=\thisrow{#2}] {\FigurePath/#1/h/unref_0.1/unref_h_0.3/order_1/outputs.txt}node[pos=0.9, pin={[pin edge=solid]90:$h$ ($p=1$)}]{};
}}}

\IfSubStr{#1}{singular2D}{%
\addplot+[ line width=1pt] table[x expr=\thisrow{nr_dof}^(#3),y expr=\thisrow{#2}] {\FigurePath/#1/h/unref_0.1/unref_h_0.3/order_2/outputs.txt} node[pos=0.95, pin={[pin edge=solid, pin distance=0.2cm]90:$h$ ($p=2$)}]{};%
}{%
\IfSubStr{#1}{atancircle2D}{%
\addplot+[ line width=1pt] table[x expr=\thisrow{nr_dof}^(#3),y expr=\thisrow{#2}] {\FigurePath/#1/h/unref_0.1/unref_h_0.3/order_2/outputs.txt} node[pos=0.98, pin={[pin edge=solid]-90:$h$ ($p=2$)}]{};%
}{%
\IfSubStr{#1}{lap2DLshape}{%
\addplot+[ line width=1pt] table[x expr=\thisrow{nr_dof}^(#3),y expr=\thisrow{#2}] {\FigurePath/#1/h/unref_0.1/unref_h_0.3/order_2/outputs.txt} node[pos=0.1, pin={[pin edge=solid]0:$h$ ($p=2$)}]{};%
}{%
\IfSubStr{#1}{Helm2DGOA}{%
\addplot+[ line width=1pt] table[x expr=\thisrow{nr_dof}^(#3),y expr=\thisrow{#2}] {\FigurePath/#1/h/unref_0.1/unref_h_0.3/order_2/outputs.txt} node[pos=0.95, pin={[pin edge=solid]180:$h$ ($p=2$)}]{};%
}{%
\IfSubStr{#1}{DiffConv2D}{%
\addplot+[ line width=1pt] table[x expr=\thisrow{nr_dof}^(#3),y expr=\thisrow{#2}] {\FigurePath/#1/h/unref_0.1/unref_h_0.3/order_2/outputs.txt} node[pos=0.1, pin={[pin edge=solid]0:$h$ ($p=2$)}]{};%
}{%
\addplot+[ line width=1pt] table[x expr=\thisrow{nr_dof}^(#3),y expr=\thisrow{#2}] {\FigurePath/#1/h/unref_0.1/unref_h_0.3/order_2/outputs.txt} node[pos=0.9, pin={[pin edge=solid]-90:$h$ ($p=2$)}]{};%
}%
}
}
}
}



%\addlegendentry{$h$ ($p=2$)}

%\addplot+[ line width=1pt] table[x expr=\thisrow{nr_dof},y expr=\thisrow{#2}] {\FigurePath/#1/p/unref_0.2/outputs.txt};
%\addlegendentry{$p$}

%\begin{scope}[yshift=0.5em]
%\ifthenelse{#2 = 1}{
%\addplot[color=red, thin] [domain=\maxdof/4:\maxdof, samples=10]{exp(-#3*x^(1/#2))*\minerror/exp(-#3*\maxdof^(1/#2))} node [pos=0.3](pin){};
%%,pin={60:$e^{-#3N}$}]{};
%}%
%{
%\addplot[color=red, thin] [domain=1:\maxdof, samples=10]{exp(-#3*x^(1/#2))*\minerror/exp(-#3*\maxdof^(1/#2))} node [pos=0.3](pin){};
%}
%
%\end{scope}



\end{axis}
\end{tikzpicture}
}


\pgfplotsset{select coords between index/.style 2 args={
    x filter/.code={
        \ifnum\coordindex<#1\def\pgfmathresult{}\fi
        \ifnum\coordindex>#2\def\pgfmathresult{}\fi
    }
}}

\newcommand{\errorplotlogUnrefBorrar}[2]{
%
\renewcommand{\DataPath}{\FigurePath/#1}
%
\findmax{\DataPath/outputs.txt}{nr_dof}{\maxdof}
\findmin{\DataPath/outputs.txt}{nr_dof}{\mindof}
%\lastelement{\DataPath/outputs.txt}{nr_dof}{\lastdof}
%\lastelement{\DataPath/outputs.txt}{Error}{\lasterror}
\findmaxDbl{\DataPath/outputs.txt}{Error}{\maxerror}
\findminDbl{\DataPath/outputs.txt}{Error}{\minerror}
%
%\tikzset{/tikz/external/export next=false}
%
%
\begin{tikzpicture}
%\pgfplotsset{ log y ticks with fixed point,}

\begin{axis}[name=mainerrorplot,
xlabel={Number of DoF},
%x dir=reverse,
ylabel=Relative error in the QoI (\%),
ymode=log,
xmode=log,
%extra y ticks={1e3},
ymin=1e-7,
ymax=1e3,
xmin=\mindof,
xmax=\maxdof,
% extra y tick labels={\pgfmathprintnumber[std, precision=2]{\minerror},\pgfmathprintnumber[std, precision=2]{\maxerror}},
%yticklabel pos=right,
width=0.9\textwidth,height=0.85\plotwidthmesh,
ylabel near ticks,
xlabel near ticks,
enlargelimits=true,
legend style={draw=black,fill=white,legend cell align=left, at={(0.5,1.01)}, anchor=south},
legend columns=-1
]

\addplot+[ line width=1pt, select coords between index={0}{#2}] table[x expr=\thisrow{nr_dof},y expr=\thisrow{Error}] {\FigurePath/#1/outputs.txt};

\end{axis}
\end{tikzpicture}
}


%:\errorplotlogUnref
\newcommand{\errorplotlogUnref}[2]{
%


\renewcommand{\DataPath}{\FigurePath/#1}


%
\findmax{\DataPath/outputs.txt}{nr_dof}{\maxdof}
\findmin{\DataPath/outputs.txt}{nr_dof}{\mindof}
%\lastelement{\DataPath/outputs.txt}{nr_dof}{\lastdof}
%\lastelement{\DataPath/outputs.txt}{Error}{\lasterror}
\findmaxDbl{\DataPath/outputs.txt}{Error}{\maxerror}
\findminDbl{\DataPath/outputs.txt}{Error}{\minerror}
%
%\tikzset{/tikz/external/export next=false}
%
%
\begin{tikzpicture}
%\pgfplotsset{ log y ticks with fixed point,}

\begin{axis}[name=mainerrorplot,
xlabel={Number of dof},
%x dir=reverse,
ylabel=Relative error in \%,
ymode=log,
xmode=log,
%extra y ticks={\minerror,\maxerror},
ymin=\minerror,
ymax=\maxerror,
xmin=\mindof,
xmax=\maxdof,
% extra y tick labels={\pgfmathprintnumber[std, precision=2]{\minerror},\pgfmathprintnumber[std, precision=2]{\maxerror}},
%yticklabel pos=right,
width=0.9\textwidth,height=0.75\plotwidthmesh,
ylabel near ticks,
xlabel near ticks,
enlargelimits=true,
legend style={draw=black,fill=white,legend cell align=left, at={(0.5,1.01)}, anchor=south},
legend columns=-1
]

\addplot+[ line width=1pt, select coords between index={0}{#2}] table[x expr=\thisrow{nr_dof},y expr=\thisrow{Error}] {\FigurePath/#1/outputs.txt};

\end{axis}
\end{tikzpicture}
}


%:\plothpDsol
\newcommand{\plothpDsol}[3]{%
\renewcommand{\DataPath}{\FigurePath/#1}
\pgfplotsset{%
    colormap={paraview}{rgb=(0.231373, 0.298039, 0.752941) rgb=(0.865003, 0.865003, 0.865003) rgb=(0.705882, 0.0156863, 0.14902)}%
}%    
\pgfplotsset{colormap name={paraview}}%

%\getelemdbl{\DataPath/#3_ColorbarValues.dat}{min_value}{#2}{\scaleMin}
%\getelemdbl{\DataPath/#3_ColorbarValues.dat}{max_value}{#2}{\scaleMax}
%\getelemdbl{\DataPath/#3_ColorbarValues.dat}{mean_value}{#2}{\scaleMean}

\getelemdbl{\DataPath/#3_ColorbarValues.dat}{min_value}{0}{\scaleMin}
\getelemdbl{\DataPath/#3_ColorbarValues.dat}{max_value}{0}{\scaleMax}
\getelemdbl{\DataPath/#3_ColorbarValues.dat}{mean_value}{0}{\scaleMean}



%\tikzset{/tikz/external/export next=false}

\begin{tikzpicture}
\begin{axis}[name=master,
width=\plotwidthmesh,height=0.95\plotwidthmesh,
enlargelimits=false,
xlabel=$x$,
ylabel=$y$,
xlabel near ticks,
ylabel near ticks,
axis line style={draw=none},
tick style={draw=none},
% hide axis,
colorbar horizontal,
colorbar style={%
	xtick={\scaleMin,\scaleMean,\scaleMax},
        title={Absolute value of the solution},
       at={(master.above north west)},anchor=south west,
	yshift=2pt,
     title style={
       yshift=4pt
         },
         yshift=0.2em,
      xticklabel pos=upper,
            },      
%point meta min=\minp,
point meta min=\scaleMin,
point meta max=\scaleMax,
]
%
\IfSubStr{#1}{square}{%
 \addplot [forget plot] graphics[xmin=-1,xmax=1,ymin=-1,ymax=1] {\DataPath/coarse_mesh_fine_#2_#3.png};
}{
%\IfSubStr{#1}{Helm}{%
% \addplot [forget plot] graphics[xmin=0,xmax=1,ymin=0,ymax=1] {\DataPath/coarse_mesh_fine_#2_#3.png};
%}{
%\IfSubStr{#1}{DiffConv}{%
% \addplot [forget plot] graphics[xmin=0,xmax=1,ymin=0,ymax=1] {\DataPath/coarse_mesh_fine_#2_Direct_solution_real.png};
%}{
%\IfSubStr{#1}{lap}{%
% \addplot [forget plot] graphics[xmin=0,xmax=1,ymin=0,ymax=1] {\DataPath/coarse_mesh_fine_#2_Direct_solution_real.png};
%}{
%\IfSubStr{#1}{DiffConv2DGOA}{%
% \addplot [forget plot] graphics[xmin=0,xmax=1,ymin=0,ymax=1] {\DataPath/coarse_mesh_fine_#2_Direct_solution_real.png};
%}{
 \addplot [forget plot] graphics[xmin=0,xmax=1,ymin=0,ymax=1] {\DataPath/coarse_mesh_fine_#2_#3.png};
% }
% }
% }
% }
 }
 
\end{axis}
\end{tikzpicture}
}

%:\plothpDmesh
\newcommand{\plothpDmesh}[3]{%
\renewcommand{\DataPath}{\FigurePath/#1}

\lastelement{\DataPath/outputs.txt}{MaxP}{\maxp}
\lastelement{\DataPath/outputs.txt}{MinP}{\minp}
%\tikzset{/tikz/external/export next=false}
\pgfplotsset{colormap/YlOrRd}
\begin{tikzpicture}
\begin{axis}[name=master,
width=\textwidth,height=\textwidth,
enlargelimits=false,
xlabel=$x$,
ylabel=$y$,
xlabel near ticks,
ylabel near ticks,
axis line style={draw=none},
tick style={draw=none},
 hide axis,
colorbar horizontal,
colorbar style={%
	xtick={1,3,...,\maxp},
    title={Order of approximation},
    at={(master.above north west)},anchor=south west,
	yshift=1em,
    title style={
       yshift=2pt
         },
     xticklabel pos=upper,
     },      
%point meta min=\minp,
point meta min=1,
point meta max=\maxp,
]
%
\IfSubStr{#1}{square}{%
 \addplot [forget plot] graphics[xmin=-1,xmax=1,ymin=-1,ymax=1] {\DataPath/mesh_#3_#2.png};
}{
 \addplot [forget plot] graphics[xmin=0,xmax=1,ymin=0,ymax=1] {\DataPath/mesh_#3_#2.png};
 }
 
\end{axis}
\end{tikzpicture}
}
%:\plothpDmeshGIF
\newcommand{\plothpDmeshGIF}[4]{%
\renewcommand{\DataPath}{\FigurePath/#1}

\lastelement{\DataPath/outputs.txt}{MaxP}{\maxp}
\lastelement{\DataPath/outputs.txt}{MinP}{\minp}
%\tikzset{/tikz/external/export next=false}
\pgfplotsset{colormap/YlOrRd}

\begin{animateinline}[autoplay]{#4}
\multiframe{#3}{iter=000+1}{
\begin{tikzpicture}
\begin{axis}[name=master,
width=\textwidth,height=\textwidth,
enlargelimits=false,
xlabel=$x$,
ylabel=$y$,
xlabel near ticks,
ylabel near ticks,
 hide axis,
colorbar horizontal,
colorbar style={%
	xtick={1,3,...,\maxp},
    title={Order of approximation},
    at={(master.above north west)},anchor=south west,
	yshift=1em,
    title style={
       yshift=2pt
         },
     xticklabel pos=upper,
     },      
%point meta min=\minp,
point meta min=1,
point meta max=\maxp,
]
%
\FormatIntegerThreeDigit{\iter}{\iterformatedthree}
\IfSubStr{#1}{square}{%
 \addplot [forget plot] graphics[xmin=-1,xmax=1,ymin=-1,ymax=1] {\DataPath/mesh_#2_\iterformatedthree.png};
}{
 \addplot [forget plot] graphics[xmin=0,xmax=1,ymin=0,ymax=1] {\DataPath/mesh_#2_\iterformatedthree.png};
 }
 
\end{axis}
\end{tikzpicture}
}
\end{animateinline}
}


%:\plothDmesh
\newcommand{\plothDmesh}[3]{%
\renewcommand{\DataPath}{\FigurePath/#1}
%
%\lastelement{\DataPath/outputs.txt}{MaxP}{\maxp}
%\lastelement{\DataPath/outputs.txt}{MinP}{\minp}
%
%\pgfplotsset{colormap/YlOrRd}
%\tikzset{/tikz/external/export next=false}
\begin{tikzpicture}
\begin{axis}[name=master,
width=\textwidth,height=\textwidth,
enlargelimits=false,
xlabel=$x$,
ylabel=$y$,
xlabel near ticks,
ylabel near ticks,
hide axis,
]
%

\IfSubStr{#1}{square}{%
 \addplot [forget plot] graphics[xmin=-1,xmax=1,ymin=-1,ymax=1] {\DataPath/grid_#2.png};
}{
 \addplot [forget plot] graphics[xmin=0,xmax=1,ymin=0,ymax=1] {\DataPath/grid_#2.png};
 }
 \addplot [forget plot, opacity=0.3] graphics[xmin=0,xmax=1,ymin=0,ymax=1] {\FigurePath/#3};


\end{axis}
\end{tikzpicture}%
}
\newcommand{\plothpsol}[2][]{

\renewcommand{\DataPath}{\FigurePath/#2/hp/p2/unref_0.1/unref_h_0.3}
\findmax{\DataPath/outputs.txt}{iter}{\endIter}

\FormatIntegerTwoDigit{\endIter}{\endIterformated}
\FormatIntegerThreeDigit{\endIter}{\endIterformatedthree}
%
\only<+>{%
\begin{figure}
\centering
\begin{subfigure}{\subplotwidth}
\plothpDsol{#2/hp/p2/unref_0.1/unref_h_0.3}{\endIterformated}{Solution_real}%
\caption{Solution of the direct problem}
\end{subfigure}
\begin{subfigure}{\subplotwidth}
\plothpDsol{#2/hp/p2/unref_0.1/unref_h_0.3}{\endIterformated}{Adjoint_solution_real}%
\caption{Solution of the adjoint problem}
\end{subfigure}


%  \begin{subfigure}{\subplotwidth}
%\plothpDsol{#2/hp/p2/unref_0.1/unref_h_0.3}{\endIterformated}{Solution_real}%
%\caption{Numerical solution of a Helmholtz problem}
%  \end{subfigure}
%  \begin{subfigure}{\subplotwidth}
%
%\plothpDsol{#2/hp/p2/unref_0.1/unref_h_0.3}{\endIterformated}{Solution_real}%
%\caption{Numerical solution of a Helmholtz problem}
%  \end{subfigure}


\end{figure}
%  \begin{subfigure}{\subplotwidth}
%    \plotsinglesolution{#2/#1}{\endIterformated}{forward_solution_#3}{#1}%
%    \subcaption{Solution to the forward problem.}
%    \label{fig:forward_solution_#2_dir}
%  \end{subfigure}
%  \begin{subfigure}{\subplotwidth}
%    \plotsinglesolution{#2/#1}{\endIterformated}{adjoint_solution_#3}{#1}%
%    \subcaption{Solution to the adjoint problem.}
%    \label{fig:adjoint_solution_#2_adj}
%  \end{subfigure}
  
  
}%
%

}%

\newcommand{\plothpsolmeshhp}[2][]{

\renewcommand{\DataPath}{\FigurePath/#2/hp/p2/unref_0.1/unref_h_0.3}
\findmax{\DataPath/outputs.txt}{iter}{\endIter}

\FormatIntegerTwoDigit{\endIter}{\endIterformated}
\FormatIntegerThreeDigit{\endIter}{\endIterformatedthree}
%
%
{
\only<+>{%
\begin{figure}
\centering
\foreach \i in {x,y}{%
\begin{subfigure}{\subplotwidth}
\centering

\plothpDmesh{#2/hp/p2/unref_0.1/unref_h_0.3}{\endIterformatedthree}{\i}%
\subcaption{Polynomial orders in the \i-direction}
\end{subfigure}
}%
\caption{Final $hp$-adapted meshes after the Goal Oriented Adaptivity}
\end{figure}
}
%
}
}%


\newcommand{\plothpsolmesh}[2][]{

\renewcommand{\DataPath}{\FigurePath/#2/hp/p2/unref_0.1/unref_h_0.3}
\findmax{\DataPath/outputs.txt}{iter}{\endIter}

\FormatIntegerTwoDigit{\endIter}{\endIterformated}
\FormatIntegerThreeDigit{\endIter}{\endIterformatedthree}
%
\only<+>{%
\begin{figure}
\centering
\plothpDsol{#2/hp/p2/unref_0.1/unref_h_0.3}{\endIterformated}{Solution_real}%
\caption{Numerical solution of a Helmholtz problem}
\end{figure}
}%
%
{
\only<+>{%
\begin{figure}
\centering
\foreach \i in {x,y}{%
\begin{subfigure}{\subplotwidth}
\centering

\plothpDmesh{#2/hp/p2/unref_0.1/unref_h_0.3}{\endIterformatedthree}{\i}%
\subcaption{Polynomial orders in the \i-direction}
\end{subfigure}
}%
\caption{Final $hp$-adapted meshes after the Goal Oriented Adaptivity}
\end{figure}
}
%
}
}%


\newcommand{\plothunrefAdapt}[1]{
\renewcommand{\DataPath}{\FigurePath/Unref/#1/h/unref_0.1/unref_h_0.3/order_1}
\findmax{\DataPath/outputs.txt}{iter}{\endIter}

\findmax{\FigurePath/#1/hp/p2/unref_0.1/unref_h_0.3/outputs.txt}{iter}{\endIterh}

\FormatIntegerTwoDigit{\endIterh}{\endIterhformated}

\begin{center}

\begin{figure}
\foreach \i in {1,2,...,\endIter}{%
\only<+>{%
\begin{subfigure}{\subplotwidth}
\centering
%\tikzset{/tikz/external/export next=false}
\FormatIntegerThreeDigit{\i}{\iformat}
\plothDmesh{Unref/#1/h/unref_0.1/unref_h_0.3/order_1}{\iformat}{#1/hp/p2/unref_0.1/unref_h_0.3/coarse_mesh_fine_\endIterhformated_Solution_real}%
\pgfmathparse{add(\i,0)} 
\pgfmathtruncatemacro\plotindex{\pgfmathresult}
\pgfmathparse{add(\endIter,0)} 
\pgfmathtruncatemacro\lastindex{\pgfmathresult}
%\vspace{10mm}
\subcaption{Adaptive iteration \i ~of \endIter}
\end{subfigure}
\begin{subfigure}{\subplotwidth}
\centering
% % %
\errorplotlogUnref{Unref/#1/h/unref_0.1/unref_h_0.3/order_1}{\i}
\subcaption{Error in $H^{1}$-semi-norm}
\end{subfigure}
\only<1->{\caption{Adaptive process}}
}%
}
%}
\end{figure}
\end{center}
}


%:\plothpD
\newcommand{\plothpD}[2][]{

\renewcommand{\DataPath}{\FigurePath/#2/hp/p2/unref_0.1/unref_h_0.3}
\findmax{\DataPath/outputs.txt}{iter}{\endIter}

\FormatIntegerTwoDigit{\endIter}{\endIterformated}
\FormatIntegerThreeDigit{\endIter}{\endIterformatedthree}
%
\only<+>{%
\ifthenelse{\equal{#1}{GOA}}{%
\begin{figure}
\centering
\begin{subfigure}{\subplotwidth}
\plothpDsol{#2/hp/p2/unref_0.1/unref_h_0.3}{\endIterformated}{Direct_solution_real}%
\caption{Solution of the direct problem}
\end{subfigure}
\begin{subfigure}{\subplotwidth}
\plothpDsol{#2/hp/p2/unref_0.1/unref_h_0.3}{\endIterformated}{Adjoint_solution_real}%
\caption{Solution of the adjoint problem}
\end{subfigure}
\end{figure}
}{%
\begin{figure}
\centering
\plothpDsol{#2/hp/p2/unref_0.1/unref_h_0.3}{\endIterformated}{Solution_real}%
\caption{Solution of the \captionifthen{#2} Problem}
\end{figure}
}%
}%
%
\ifthenelse{\equal{#2}{Helm2DGOA}}{%
\only<+>{%
\begin{figure}
\centering
\foreach \i in {x,y}{%
\begin{subfigure}{\subplotwidth}
\centering

\plothpDmesh{Helm2DGOANon/hp/p2/unref_0.1/unref_h_0.3}{\endIterformatedthree}{\i}%
\subcaption{Polynomial orders in the \i-direction}
\end{subfigure}
}%
\caption{Final $hp$-adapted meshes in \alert{energy-norm adaptivity}}
\end{figure}
}
\only<+>{%
\begin{figure}
\centering
\foreach \i in {x,y}{%
\begin{subfigure}{\subplotwidth}
\centering

\plothpDmesh{#2/hp/p2/unref_0.1/unref_h_0.3}{\endIterformatedthree}{\i}%
\subcaption{Polynomial orders in the \i-direction}
\end{subfigure}
}%
\caption{Final $hp$-adapted meshes in \alert{goal-oriented adaptivity}}
\end{figure}
}
%	
}
{
\only<+>{%
\begin{figure}
\centering
\foreach \i in {x,y}{%
\begin{subfigure}{\subplotwidth}
\centering

\plothpDmesh{#2/hp/p2/unref_0.1/unref_h_0.3}{\endIterformatedthree}{\i}%
\subcaption{Polynomial orders in the \i-direction}
\end{subfigure}
}%
\caption{Final $hp$-adapted meshes}
\end{figure}
}
%
}
\only<+>{%
\begin{figure}
\errorplotlog{#2}{Error}%
%
\caption{Evolution of the error in \captionifthenlegend{#2}}
\end{figure}
}%
}%

%:\plothDunref
\newcommand{\plothDunref}[1]{
\renewcommand{\DataPath}{\FigurePath/Unref/#1/h/unref_0.1/unref_h_0.3/order_1}
\findmax{\DataPath/outputs.txt}{iter}{\endIter}

\findmax{\FigurePath/#1/hp/p2/unref_0.1/unref_h_0.3/outputs.txt}{iter}{\endIterh}

\FormatIntegerTwoDigit{\endIterh}{\endIterhformated}

\begin{center}
\begin{figure}
\centering
\only<+>{%
\plothpDsol{#1/hp/p2/unref_0.1/unref_h_0.3}{\endIterhformated}{Solution_real}%
\caption{Solution with a strong gradient}
% % %
}
\end{figure}

\begin{figure}
\foreach \i in {1,2,...,\endIter}{%
\only<+>{%
\begin{subfigure}{\subplotwidth}
\centering
%\tikzset{/tikz/external/export next=false}
\FormatIntegerThreeDigit{\i}{\iformat}
\plothDmesh{Unref/#1/h/unref_0.1/unref_h_0.3/order_1}{\iformat}{#1/hp/p2/unref_0.1/unref_h_0.3/coarse_mesh_fine_\endIterhformated_Solution_real}%
\pgfmathparse{add(\i,0)} 
\pgfmathtruncatemacro\plotindex{\pgfmathresult}
\pgfmathparse{add(\endIter,0)} 
\pgfmathtruncatemacro\lastindex{\pgfmathresult}
%\vspace{10mm}
\subcaption{Adaptive iteration \i ~of \endIter}
\end{subfigure}
\begin{subfigure}{\subplotwidth}
\centering
% % %
\errorplotlogUnref{Unref/#1/h/unref_0.1/unref_h_0.3/order_1}{\i}
\subcaption{Error in $H^{1}$-semi-norm}
\end{subfigure}
\only<2->{\caption{Adaptive process}}
}%
}
%}
\end{figure}
\end{center}
}


\newcommand{\plothpfem}[2][]{

  \renewcommand{\DataPath}{\FigurePath/#2/hp/p2/unref_0.1/unref_h_0.3}%\PbName/hp/...
  \findmax{\DataPath/outputs.txt}{iter}{\endIter}

  \FormatIntegerThreeDigit{\endIter}{\endIterformated}

  \foreach \idim in {X,Y}{
      \begin{subfigure}{\subplotwidth}
        \plothpmeshfem{#2/hp/p2/unref_0.1/unref_h_0.3}{\endIterformated}{\idim}
        \caption{\ifthenelse{\equal{\idim}{X}}{Initial $hp$-mesh}{Final $hp$-adapted mesh}}
        %\label{fig:#2_\idim_final}
      \end{subfigure}
    }
}

\newcommand{\plothpmeshfem}[3]{
  \renewcommand{\DataPath}{\FigurePath/#1}

  \lastelement{\DataPath/outputs.txt}{MaxP}{\maxp}
  \lastelement{\DataPath/outputs.txt}{MinP}{\minp}

  %\pgfplotsset{colormap/jet} %Barra de color
  \pgfplotsset{colormap/YlOrRd} %Barra de color
  %\tikzset{/tikz/external/export next=false}
  \begin{tikzpicture}
    \begin{axis}[name=master,
        width=\textwidth,height=\textwidth,
        enlargelimits=false,
        xlabel=$x$,
        ylabel=$y$,
        xlabel near ticks,
        ylabel near ticks,
        axis line style={draw=none},
        tick style={draw=none},
        colorbar horizontal,
        colorbar style={%
            xtick={1,2,...,\maxp},
            title={Order of approximation},
            at={(master.above north west)},anchor=south west,
            yshift=2pt,
            title style={
                yshift=2pt
              },
            yshift=0.2em,
            xticklabel pos=upper,
          },
        %point meta min=\minp,
        point meta min=1,
        point meta max=\maxp,
      ]
      %
      \addplot [forget plot] graphics[xmin=0,xmax=1,ymin=0,ymax=1] {\DataPath/mesh_#3_#2.png}; %Límites del dominio de los subplots (a) y (b). #3 X o Y, #2 por ejemplo 015

    \end{axis}
  \end{tikzpicture}
}

\newcommand{\plothpDmat}[2][]{

  \renewcommand{\DataPath}{\FigurePath/#2/hp/p2/unref_0.1/unref_h_0.3}
  \findmax{\DataPath/outputs.txt}{iter}{\endIter}

  \FormatIntegerTwoDigit{\endIter}{\endIterformated}
  \FormatIntegerThreeDigit{\endIter}{\endIterformatedthree}
  %
  \plothpsolsol{#2/hp/p2/unref_0.1/unref_h_0.3}{\endIterformated}{materials}{#1}%
}

\newcommand{\plothpsolsol}[4]{%
  \renewcommand{\DataPath}{\FigurePath/#1}

  \IfSubStr{#4}{unitsquare}
  {   	\pgfplotsset{%
      colormap={paraview}{rgb=(0.231373, 0.298039, 0.752941) rgb=(0.865003, 0.865003, 0.865003) rgb=(0.705882, 0.0156863, 0.14902)}%
    }%
    \pgfplotsset{colormap name={paraview}}%

    \getelemdbl{\DataPath/#3_ColorbarValues.dat}{min_value}{0}{\scaleMin}
    \getelemdbl{\DataPath/#3_ColorbarValues.dat}{max_value}{0}{\scaleMax}
    \getelemdbl{\DataPath/#3_ColorbarValues.dat}{mean_value}{0}{\scaleMean}

    %\tikzset{/tikz/external/export next=false}
    \begin{tikzpicture}
      \begin{axis}[name=master,
          width=\textwidth*0.5,height=\textwidth*0.5,
          enlargelimits=false,
          xlabel=$x$,
          ylabel=$y$,
          xlabel near ticks,
          ylabel near ticks,
          colorbar horizontal,
          colorbar style={%
              xtick={\scaleMin,\scaleMean,\scaleMax},
              title={Value of the solution},
              at={(master.above north west)},anchor=south west,
              yshift=2pt,
              title style={
                  yshift=5pt%yshift=2pt
                },
              yshift=0.2em,
              xticklabel pos=upper,
            },
          %point meta min=\minp,
          point meta min=\scaleMin,
          point meta max=\scaleMax,
        ]
        %
        \addplot [forget plot] graphics[xmin=0,xmax=1,ymin=0,ymax=1] {\DataPath/coarse_mesh_fine_#2_#3.png};
      \end{axis}
    \end{tikzpicture}}
  {%else
    \pgfplotsset{%
      colormap={paraview}{rgb=(0.05, 1.0, 1.0) rgb=(1.0, 0.0, 0.0) rgb=(0.0, 1.0, 0.0) rgb=(0.0, 0.0, 1.0)}%
    }
    %
    \pgfplotsset{colormap name={paraview}}%
    \findmax{\DataPath/#3.txt}{uno}{\matuno}
    \findmax{\DataPath/#3.txt}{dos}{\matdos}
    \findmax{\DataPath/#3.txt}{tres}{\mattres}
    \findmax{\DataPath/#3.txt}{cuatro}{\matcuatro}

    %\tikzset{/tikz/external/export next=false}
    \begin{tikzpicture}
      \begin{axis}[name=master,
          width=\textwidth*0.5,height=\textwidth*0.5,
          enlargelimits=false,
          %xlabel=$x$,
          %ylabel=$y$,
          %xlabel near ticks,
          %ylabel near ticks,
          axis line style={draw=none},
          tick style={draw=none},
          xticklabels={},
          yticklabels={},
          colorbar horizontal,
          colorbar sampled,
          colormap access=piecewise constant,
          colorbar style={
              samples=4,
              xtick=data,%={0,1,2,3},
              xticklabels={0.01,1,10,1000},
              xticklabel style={xshift=\textwidth*0.5/10},
              title={Diffusivity of the materials},
              at={(master.above north west)},anchor=south west,
              yshift=5pt,
              title style={
                  yshift=5pt%yshift=2pt
                },
              yshift=0.2em,
              xticklabel pos=upper,
            },
          %point meta min=\minp,
          % point meta min=\matuno,
          % point meta max=\matcuatro,
        ]
        %
        \addplot [forget plot] graphics[xmin=0,xmax=1,ymin=0,ymax=1] {\DataPath/#3_#2.png};

      \end{axis}
    \end{tikzpicture}}%
}

\newcommand{\plothplevels}[2][]{

  \renewcommand{\DataPath}{\FigurePath/#2/hp/p2/unref_0.1/unref_h_0.3}%\PbName/hp
  \findmax{\DataPath/outputs.txt}{iter}{\endIter}
  %\findmin{\DataPath/outputs.txt}{iter}{\initIter} %\plothplevel

  \FormatIntegerThreeDigit{\endIter}{\endIterformated}
  \foreach \idim in {z}{
      \begin{subfigure}{\subplotwidth}
        \plothplevel{#2/hp/p2/unref_0.1/unref_h_0.3}{\endIterformated}{\idim}
        %\caption{Final $hp$-adapted mesh with polynomial orders in the $\idim$-direction}
      \end{subfigure}
    }

  %\pgfmathparse{subtract(\endIter,0)}
  %\pgfmathtruncatemacro\endIterIndex{\pgfmathresult}

  %\getelem{\DataPath/outputs.txt}{nr_dof}{\endIter}{\nrdofHpEnd}
  %\getelemdbl{\DataPath/outputs.txt}{Error_H1}{\endIter}{\errorHpEnd}
}

\newcommand{\plothplevel}[3]{
  \renewcommand{\DataPath}{\FigurePath/#1}

  \lastelement{\DataPath/outputs.txt}{MaxP}{\maxp}
  \lastelement{\DataPath/outputs.txt}{MinP}{\minp}

  %\pgfplotsset{colormap/jet} %Barra de color
  \pgfplotsset{colormap/YlOrRd} %Barra de color
  %\tikzset{/tikz/external/export next=false}
  \begin{tikzpicture}
    \begin{axis}[name=master,
        width=\textwidth,height=\textwidth,
        enlargelimits=false,
        %xlabel=$x$,
        %ylabel=$y$,
        %xlabel near ticks,
        %ylabel near ticks,
        axis line style={draw=none},
        tick style={draw=none},
        xticklabels={},
        yticklabels={},
        colorbar horizontal,
        colorbar style={%
            xtick={1,2,...,\maxp},
            title={Order of approximation},
            at={(master.above north west)},anchor=south west,
            yshift=5pt,
            title style={
                yshift=2pt
              },
            yshift=0.2em,
            xticklabel pos=upper,
          },
        %point meta min=\minp,
        point meta min=1,
        point meta max=\maxp,
      ]
      %
      \addplot [forget plot] graphics[xmin=0,xmax=1,ymin=0,ymax=1] {\DataPath/boxes/box_mesh_#3_#2.png};

    \end{axis}
  \end{tikzpicture}
}
% ==========
%\magomate
% ==========
\newcommand{\magomate}[3][]{
  \renewcommand{\DataPath}{\FigurePath/#2/norm0}
%  \findmax{\DataPath/outputs.txt}{iter}{\endIter}
%  \FormatIntegerThreeDigit{\endIter}{\endIterformated}
% \begin{center}
%  \foreach \idim in {X}{
%      \begin{subfigure}[t]{0.45\textwidth}
%    \lefthpmesh{#2/norm0}{\endIterformated}{\idim}
%    {\caption{Final $hp$-adapted mesh with $p$ in the \ifthenelse{\equal{\idim}{X}}{$x$}{$y$}-direction.}
%    \label{fig:#2_\idim_final}}
%  \end{subfigure}
%  } 
%  ~
%  \foreach \idim in {Y}{
%  \begin{subfigure}[t]{0.45\textwidth}
%   \righthpmesh{#2/norm0}{\endIterformated}{\idim}
%    {\caption{Final $hp$-adapted mesh with $p$ in the \ifthenelse{\equal{\idim}{X}}{$x$}{$y$}-direction.}
%    \label{fig:#2_\idim_final}}
%  \end{subfigure}
%  }
%   \end{center}
  \findmin{\FigurePath/#2/norm0/outputs.txt}{iter}{\firstIter}
  \FormatIntegerTwoDigit{\firstIter}{\firstIterformated} 
  
  %\begin{subfigure}[t]{0.45\textwidth}
   %\centering
   \materials{#2/norm0}{\firstIterformated}{materials_real}{mago}
    %{\caption{Values for the different materials in the domain.}}
  %\end{subfigure}
%  \begin{subfigure}[t]{0.45\textwidth}
%   \centering
%   \errormago[#1]{#2}{#3}
%    {\caption{Evolution of \captionifthenproblem{#3} in the adaptive process.}
%    \label{fig:error_#2}}
%  \end{subfigure}
}
% =======
% \materials
% =======
\newcommand{\materials}[4]{%
  \renewcommand{\DataPath}{\FigurePath/#1}
  
    	
      %
}

%% =======
%%\plotskew
%% =======
\newcommand{\plotskew}[2][]{

  \begin{subfigure}[t]{0.4\textwidth}
    \centering
    \vspace{-5.25cm}
    \skewleft{#2}
    \vspace{0.90cm}
    {\caption{Domain illustration ($5 \times 5$ grid)}}
  \end{subfigure}
  ~
 \begin{subfigure}[t]{0.35\textwidth}
    \centering
    \skewright{#2}
    {\caption{$\tcr{\boldsymbol{\sigma(x)}}$ varying 4 orders of magnitude.}}
  \end{subfigure}
}
%% ======
%%\skewleft
%% ======
\newcommand{\skewleft}[2][]{

	%\tikzset{/tikz/external/export next=false}
	
	\begin{tikzpicture}[x=0.45cm,y=0.45cm]
			\node(origin) at (0,0) {};
			\node(end) at (10,10) {};
			\node(f_int_origin) at (0.5,0.5) {};
			\node(f_int_end) at (1.5,1.5) {};
			\node(l_int_origin) at (8.5,8.5) {};
			\node(l_int_end) at (9.5,9.5) {};
			
			\draw [draw=blue,fill=white,thick] (origin) rectangle (end);
			
			\draw[] ($(origin)+(2,0)$) -- ($(origin)+(2,10)$);
			\draw[] ($(origin)+(4,0)$) -- ($(origin)+(4,10)$);
			\draw[] ($(origin)+(6,0)$) -- ($(origin)+(6,10)$);
			\draw[] ($(origin)+(8,0)$) -- ($(origin)+(8,10)$);
			
			\draw[] ($(origin)+(0,2)$) -- ($(origin)+(10,2)$);
			\draw[] ($(origin)+(0,4)$) -- ($(origin)+(10,4)$);
			\draw[] ($(origin)+(0,6)$) -- ($(origin)+(10,6)$);
			\draw[] ($(origin)+(0,8)$) -- ($(origin)+(10,8)$);
			
			\draw [draw=black,fill=white] (f_int_origin) rectangle (f_int_end) node[pos=0.5] {$\Omega_f$};
			\draw [draw=black,fill=white] (l_int_origin) rectangle (l_int_end) node[pos=0.5] {$\Omega_l$};
		\end{tikzpicture}
}
%% =======
%%\skewright
%% =======
\newcommand{\skewright}[2][]{
	\renewcommand{\DataPath}{\FigurePath/#2/norm0}
	
	\findmin{\FigurePath/#2/norm0/outputs.txt}{iter}{\firstIter}
  	\FormatIntegerTwoDigit{\firstIter}{\firstIterformated} 
  
	\getelemdbl{\DataPath/mago_ColorbarValues.dat}{min_value}{0}{\scaleMin}
  	\getelemdbl{\DataPath/mago_ColorbarValues.dat}{max_value}{0}{\scaleMax}
  	\getelemdbl{\DataPath/mago_ColorbarValues.dat}{mean_value}{0}{\scaleMean}
  
  \pgfplotsset{%
    colormap={paraview}{rgb=(0.231373, 0.298039, 0.752941) rgb=(0.865003, 0.865003, 0.865003) rgb=(0.705882, 0.0156863, 0.14902)}%
  }%
  \pgfplotsset{colormap name={paraview}}%
  
  \begin{tikzpicture}
      \begin{axis}[name=master,
          width=\textwidth,height=\textwidth,
          enlargelimits=false,
          xlabel=$x$,
          ylabel=$y$,
          xlabel near ticks,
          ylabel near ticks,
          colorbar horizontal,
          colorbar style={%
              xtick={0.1,500.05,1000},
              title={Value of $\sigma(x)$ (log scale)},
              at={(master.above north west)},anchor=south west,
              yshift=2pt,
              title style={
                  yshift=5pt
                },
              yshift=0.2em,
              xticklabel pos=upper,
            },
          point meta min=0.1,
          point meta max=1000,
        ]
        
       \addplot [forget plot] graphics[xmin=0,xmax=1,ymin=0,ymax=1] {\DataPath/mago_\firstIterformated_materials_real.png}; 

      \end{axis}
    \end{tikzpicture}
}

%%%%%%%%%%%%%%%%%%%%%%%%%%%%%%%%%%%%%%%%%%%%%%%%%%%%%%%% OLD
% ==============
% elemental routines
% ==============
\newcommand{\FormatIntegerTwoDigit}[2]{
  \pgfmathsetbasenumberlength{2}
  \pgfmathbasetodec#2{#1}{10}
}

\newcommand{\FormatIntegerThreeDigit}[2]{
  \pgfmathsetbasenumberlength{3}
  \pgfmathbasetodec#2{#1}{10}
}

\newcommand{\getelem}[4]{
  \pgfplotstablegetelem{#3}{#2}\of{#1}%
  \pgfmathtruncatemacro#4{\pgfplotsretval}
  \pgfplotstableclear{\datatable}
}


\newcommand{\getelemdbl}[4]{
  \pgfplotstablegetelem{#3}{#2}\of{#1}%
  \pgfmathsetmacro#4{\pgfplotsretval}
  \pgfplotstableclear{\datatable}
}

\newcommand{\findmax}[3]{
  \pgfplotstableread{#1}{\datatable}
  \pgfplotstablesort[sort key={#2},sort cmp={float >}]{\sorted}{\datatable}%
  \pgfplotstablegetelem{0}{#2}\of{\sorted}%
  \pgfmathtruncatemacro#3{\pgfplotsretval}
  \pgfplotstableclear{\datatable}
}

\newcommand{\findmin}[3]{
  \pgfplotstableread{#1}{\datatable}
  \pgfplotstablesort[sort key={#2},sort cmp={float <}]{\sorted}{\datatable}%
  \pgfplotstablegetelem{0}{#2}\of{\sorted}%
  \pgfmathtruncatemacro#3{\pgfplotsretval}
  \pgfplotstableclear{\datatable}
}

\newcommand{\findmaxDbl}[3]{
  \pgfplotstableread{#1}{\datatable}
  \pgfplotstablesort[sort key={#2},sort cmp={float >}]{\sorted}{\datatable}%
  \pgfplotstablegetelem{0}{#2}\of{\sorted}%
  \pgfmathsetmacro#3{\pgfplotsretval}
  \pgfplotstableclear{\datatable}
}

\newcommand{\lastelement}[3]{
  \pgfplotstableread{#1}{\datatable}
  \pgfplotstablegetrowsof{\datatable} 
  \pgfmathtruncatemacro{\rows}{\pgfplotsretval}
  \pgfmathtruncatemacro{\lastrow}{\rows-1}
  \pgfplotstablegetelem{\lastrow}{#2}\of{\datatable}
  \pgfmathsetmacro#3{\pgfplotsretval}
  \let#3=\pgfplotsretval
  \pgfplotstableclear{\datatable}
}

\newcommand{\findminDbl}[3]{
  \pgfplotstableread{#1}{\datatable}
  \pgfplotstablesort[sort key={#2},sort cmp={float <}]{\sorted}{\datatable}%
  \pgfplotstablegetelem{0}{#2}\of{\sorted}%
  \pgfmathsetmacro#3{\pgfplotsretval}
  \let#3=\pgfplotsretval
  \pgfplotstableclear{\datatable}
}

\pgfplotsset{
  log x ticks with fixed point/.style={
      xticklabel={
          \pgfkeys{/pgf/fpu=true}
          \pgfmathparse{exp(\tick)}%
          \pgfmathprintnumber[fixed relative, precision=3]{\pgfmathresult}
          \pgfkeys{/pgf/fpu=false}
        }
    },
  log y ticks with fixed point/.style={
      yticklabel={
          \pgfkeys{/pgf/fpu=true}
          \pgfmathparse{exp(\tick)}%
          \pgfmathprintnumber[fixed relative, precision=3]{\pgfmathresult}
          \pgfkeys{/pgf/fpu=false}
        }
    }
}

\pgfplotscreateplotcyclelist{custom list style}{%
  color=red, solid, every mark/.append style={solid},mark size=4.0pt,mark=o\\%
  color=blue, solid, every mark/.append style={solid},mark size=3.0pt,mark=square\\%
  color=gray!20!black!40, solid, every mark/.append style={solid}, mark size=2.5pt, mark=triangle\\%
  color=black, solid, every mark/.append style={solid},mark=diamond*\\%
  color=black!50!red, solid, every mark/.append style={solid}, mark=otimes*\\%
  color=black!50!yellow, loosely dashed, every mark/.append style={solid},mark=*\\%
  color=black!50!blue,densely dashed, every mark/.append style={solid},mark=square*\\%
  color=black!50!magenta,dashdotted, every mark/.append style={solid},mark=otimes*\\%
  color=black!50!orange,dashdotdotted, every mark/.append style={solid},mark=star\\%
  color=black!50!cyan, densely dashdotted,every mark/.append style={solid},mark=diamond*\\%
}

\pgfplotsset{%
  every axis plot/.append style= {line width=2pt},
  cycle list name=custom list style,
}
