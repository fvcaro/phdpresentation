%%%%%%%%%%%%%%%%%%%%%%%%%%%%%%%%%%%%%%%%%%%%%%%%%%%%%%%%%%%%%%%%%%%
%%%%%%%%%%%%%%%%%%%%%%%%%%%%%%%%%%%%%%%%%%%%%%%%%%%%%%%%%%%%%%%%%%%
\begin{frame}{Deadlines}

Versión 1: 1 Abril\\
Yo lo corrijo para el 4 de Abril.
\vspace{2cm}

Versión 2: 13 Abril\\
Yo lo corrijo para el 16 de Abril.

\end{frame}
%%%%%%%%%%%%%%%%%%%%%%%%%%%%%%%%%%%%%%%%%%%%%%%%%%%%%%%%%%%%%%%%%%%
\begin{frame}{Contenido}

\begin{itemize}
\item 1 - Title, coauthors, etc. (Javi)
\item 2 :  Geophysical Applications: CO2 storage, Hydrogen storage, Earthquakes, Oil \& Gas, etc. (Lena)
\item 3 : Traditional simulations: Show the cover of my book (Jon Ander)
\item 1: Javi, menciona la red www.mathdata.science, y también el proyecto mathrocks www.mathrocks.science 
\item 1 :  Mathematical Formulation: Forward and Inverse Problem (Jon Ander)
\item 2  : The problems are governed by PDEs (Jon Ander)
\item 6: Deep Learning for Inverse Problems, loss function, and why we want to solve the forward problem with Deep Learning. The Forward Problem is a Parametric PDE (Jon Ander).
\item 1  : Let´s first solve a non-parametric PDE in 1D.
\item 5 : What people do (PINNs, etc. Javi)
\item 2 : What people do (Alguna clasificación que se te ocurra basado en el super-paper que estás haciendo, Lena)
\end{itemize}
\end{frame}


%%%%%%%%%%%%%%%%%%%%%%%%%%%%%%%%%%%%%%%%%%%%%%%%%%%%%%%%%%%%%%%%%%%
\begin{frame}{Contenido}
\begin{itemize}
\item 1 : Why we want $H^1$ methods for geophysics (Ej. en 1D donde PINNs tradicionales no funcionan, Jamie).
\item 2: Problems with Integration (Javi)
\item 6 : Possible Solutions (Javi)
\item 5: Carlos Deep FEM (First normal, then parametric; brief, without details, Carlos.)
\item 7: r-adaptivity (brief, Javi)
\item 5: Deep Inf-sup + Double Ritz (Carlos, brief).
\item 15: Fourier --> $H^{-1} $(Jamie).
\item 1: Conclusions (Javi)
\item 1: Future work (Javi)
\item 1: Announcement: We are looking for Ph.D. students and Postdocs (Javi).
\end{itemize}

\end{frame}


