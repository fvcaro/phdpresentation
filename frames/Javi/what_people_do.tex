\begin{frame} {Deep Ritz Method (DRM)}

 
Minimization of the total energy:

\vspace{0.5cm}

\begin{equation*}
 \mathcal{L}_{Ritz}  := \frac{1}{2} \int_{\Omega} (\nabla u )^2 - \int_{\Omega} f \, u - \int_{\Gamma_N }g \, u
\end{equation*}

\vspace{0.5cm}

\begin{itemize}
\item Requires a symmetric and positive definite bilinear form.
\end{itemize}

\vspace{1cm}

%\beamertemplatebookbibitems
\begin{thebibliography}{1}
\bibitem{Ritz}E, W., Yu, B.: The Deep Ritz method: A Deep Learning-Based Numerical Algorithm for Solving Variational Problems. Communications in Mathematics and Statistics \textbf{6}, 1--12 (2018)
\end{thebibliography}

\end{frame}
%%%%%%%%%%%%%%%%%%%%%%%%%%%%%%%%%%%%%%%%%%%%%%%%%%%%%%%%%%%%%%%%%%%
%%%%%%%%%%%%%%%%%%%%%%%%%%%%%%%%%%%%%%%%%%%%%%%%%%%%%%%%%%%%%%%%%%%
%%%%%%%%%%%%%%%%%%%%%%%%%%%%%%%%%%%%%%%%%%%%%%%%%%%%%%%%%%%%%%%%%%%
%%%%%%%%%%%%%%%%%%%%%%%%%%%%%%%%%%%%%%%%%%%%%%%%%%%%%%%%%%%%%%%%%%%
\begin{frame} {Numerical Methods Based on Residual Minimization}

%\begin{columns}
%\begin{column}{0.5\textwidth}
%    \textbf{residual in strong-form}
%%\vspace{0.5cm}
%\begin{equation*}
%\renewcommand\arraystretch{2}
%\left\{
%\begin{array}{rll}
%r   & := \bigtriangleup u + f = 0,      & x \in \Omega,\\ 
%r_N & := \nabla u \cdot \bs{n} -g = 0, &  x  \in \Gamma_N.
%\end{array}
%\right.
%\end{equation*}
%
%\end{column}
%\begin{column}{0.5\textwidth} 
%	\textbf{residual in weak-form}
%\begin{equation*}
%\renewcommand\arraystretch{2}
%{ \everymath={\displaystyle}	
%\left\{
%\begin{array}{rll}
% R_i  & := \int_{\Omega} r \, v_i,  \\
% R_{N,i} & := \int_{\Gamma_N} r_N \,  v_i.
%\end{array}
%\right.
%}
%\end{equation*}
%
%\end{column}
%\end{columns}

\begin{equation*}
\mathcal{L}_{(\cdot)} := \sum_{i=1}^{n} \left\langle   \textcolor{blue}{\bigtriangleup u + f}, \textcolor{red}{v_i}   \right\rangle_{L^2(\Omega)} +
\omega_N \sum_{i=1}^{n_N} \left\langle  \textcolor{blue}{ \nabla u \cdot \bs{n} -g} , \textcolor{red}{ v_i}\right\rangle_{L^2(\Gamma_N)}
\end{equation*}

The selection of \textcolor{red}{test functions} determines the numerical method:

\centering
%
%
\vspace{0.2cm}
\renewcommand{\arraystretch}{1.3}
\begin{table}[]
\begin{tabular}{lll}
\rowcolor[HTML]{C0C0C0} \hline  
 Method & \textcolor{red}{Test function} &  Deep method\\ \hline \hline
Collocation & sgn(\textcolor{blue}{residual}) $\cdot$ Dirac delta & PINNs  
\\ \rowcolor[HTML]{EFEFEF}
Least-Squares & \textcolor{blue}{residual} &  DGM
\\
Petrov-Galerkin &  \begin{tabular}[c]{@{}l@{}} polynomials \\ piece-wise polynomials\end{tabular}& \begin{tabular}[c]{@{}l@{}}VPINNs \\ VarNet  \end{tabular} 
\\ \rowcolor[HTML]{EFEFEF}
\begin{tabular}[c]{@{}l@{}} Petrov-Galerkin with \\ domain decomposition \end{tabular} & \begin{tabular}[c]{@{}l@{}} polynomials over \\ each subdomain \end{tabular} &  $hp-$VPINNS 
\end{tabular}
\end{table}

\end{frame}
%%%%%%%%%%%%%%%%%%%%%%%%%%%%%%%%%%%%%%%%%%%%%%%%%%%%%%%%%%%%%%%%%%%

\begin{frame} {References:}
\bibliographystyle{splncs04}{\small \bibliography{bib/biblio_1}}
\textcolor{white}{\tiny
\cite{pinn}
\cite{dgm}
\cite{VarNET}
\cite{hp_vpinn}
}
\end{frame}